\documentclass{PRM}

\usepackage{mathtools}
\usepackage{natbib}
\usepackage{xcolor}
\usepackage{hyperref}
\usepackage{fancyhdr}
\usepackage{amsmath}            
\usepackage{amssymb}             
\usepackage{amsfonts}            
\usepackage{latexsym}            
\usepackage{amsthm}             

\newcommand{\field}[1]{\mathbb{#1}}
\newcommand{\N}{\field{N}}
\newcommand{\Z}{\field{Z}}
\newcommand{\Q}{\field{Q}}
\newcommand{\R}{\field{R}}
\newcommand{\C}{\field{C}}

\newcommand{\updim}{\overline{\dim}}
\newcommand{\lowdim}{\underline{\dim}}
\newcommand{\bari}{\bar{\imath}}
\newcommand{\smalli}{\mathtt{i}}
\newcommand{\norm}[1]{\left|\left|#1\right|\right|}
\newcommand{\diam}{\mathrm{diam}}
\newcommand{\adim}{\dim_{\mathrm{A}}}

\newcommand{\iii}{\mathbf{i}}
\newcommand{\jjj}{\mathbf{j}}

\DeclareMathOperator*{\esssup}{ess\,sup}
\DeclareMathOperator*{\essinf}{ess\,inf}

\theoremstyle{plain}
\newtheorem{thm}{Theorem}[section]     
\newtheorem{lemma}[thm]{Lemma}       
\newtheorem{cor}[thm]{Corollary}      
\newtheorem{prop}[thm]{Proposition}

\theoremstyle{definition}
\newtheorem{defn}[thm]{Definition}  
\newtheorem{quest}{Question}
\newtheorem*{quest*}{Question}

\theoremstyle{remark}
\newtheorem{huom}[thm]{Remark}   
\newtheorem{example}[thm]{Example}

\begin{document}

\title{Pointwise Assouad dimension for measures}

\author{Roope \surname{Anttila}}

\address{Research Unit of Mathematical Sciences\\  P.O.Box 8000, FI-90014, University of Oulu, Finland \email{roope.anttila@oulu.fi}}

\begin{abstract}
    We introduce a pointwise variant of the Assouad dimension for measures on metric spaces, and study its properties in relation to the global Assouad dimension. We show that, in general, the value of the pointwise Assouad dimension differs from the global counterpart, but in many classical cases, it exhibits similar exact dimensionality properties as the classical local dimension, namely it equals the global Assouad dimension almost everywhere. We also compute the Assouad dimension of invariant measures with place dependent probabilities supported on strongly separated self-conformal sets. 
    \end{abstract}


\keywords{pointwise doubling measure, pointwise Assouad dimension, quasi-Bernoulli measure, self-conformal set, place-dependent probabilities}

\classification{primary: 28A80; secondary: 28C15}

\maketitle

\section{Introduction}
Originally, the Assouad dimension was defined as a means to investigate embedding problems of metric spaces \cite{A} and it is still used today as an efficient tool in studying these kinds of problems \cite{T}. In some sense, the Assouad dimension quantifies the size of the largest local parts of the space under investigation, which provides a heuristic on why it is so effective in the study of embedding problems: If the space has large local parts, it can not be embedded into a small space. In recent years, the Assouad dimension has gained increasing attention also in fractal geometry and dimension theory. The book by Fraser \cite{F} collects these recent developments in one place and provides an introduction to the theory of Assouad dimensions in fractal geometry. As is usual in fractal geometry, the Assouad dimension of a space is closely connected to a dual notion of dimension for the measures supported on the space. This Assouad dimension of a measure has a similar intuition behind it as the Assouad dimension of a space, it gives information on the least regular parts of the measure.

There are also many notions of pointwise dimension for measures, most important being the upper and lower local dimensions. Unlike the different notions of ``global'' dimension, which are concerned with the average regularity (e.g. in the case of Hausdorff, packing and Minkowski dimensions) or extremal regularity (in the case of Assouad and lower dimensions) of the measure on its entire support, these pointwise dimensions quantify the regularity of the measure around a given point. The upper and lower local dimensions can be thought of as the pointwise analogue of the Hausdorff and packing dimensions of the measure and our aim is to provide a natural pointwise analogue of the Assouad dimension, which captures the size of the least regular \emph{scales} of the measure at a given point. Due to the similarity in flavour to the Assouad dimension of a measure, we call this dimension the \emph{pointwise Assouad dimension} of the measure (see Definition \ref{def:loc_assouad}).

This paper is structured as follows: In Section \ref{sec:prelim} we establish some notation and recall basic results concerning Assouad dimensions of sets and measures. In Section \ref{sec:pw_assouad} we introduce the pointwise Assouad dimension and discuss its properties in relation to the existing notions of dimension. In Section \ref{sec:qb-meas} we work in a general setting of quasi-Bernoulli measures supported on strongly separated self-conformal sets and prove an exact dimensionality property for the pointwise Assouad dimension of these measures. In Section \ref{sec:invmeas} we compute the Assouad dimension of a class of place dependent invariant measures and use the results of the previous section to obtain the almost sure pointwise Assouad dimension. Finally, in Section \ref{sec:ssm} we show that the exact dimensionality property holds for doubling self-similar measures, even when the strong separation condition is relaxed to the open set condition. We also establish the property for self-affine measures on Bedford-McMullen sponges satisfying the very strong separation condition.

\section{Preliminaries}\label{sec:prelim}
Unless stated otherwise, we assume that $(X,d)$ is a metric space, with no additional structure. Since we assume the metric $d$ to be fixed, we omit it from the notation and refer to $(X,d)$ simply as $X$. The rest of this section is devoted to introducing some basic notation and results related to Assouad dimensions. Readers familiar with these concepts can easily skip to Section \ref{sec:pw_assouad}. 

\subsection{Notation}
A closed ball centered at $x\in X$ and with radius $r>0$ is denoted by $B(x,r)$. The space $X$ is said to be \emph{doubling}, if any ball $B(x,r)$ can be covered by $N$ balls $B(x,\frac{r}{2})$, where the constant $N\in \N$ is independent of $x$ and $r$. Unless stated otherwise, a measure always refers to a finite Borel measure fully supported on $X$. When needed, we denote the support of $\mu$ by $\mathrm{supp}(\mu)$. If $f:X\to Y$ is a map from $X$ to another metric space $Y$, we denote the \emph{pushforward of the measure $\mu$} under the map $f$ by $f_*\mu\coloneqq\mu\circ f^{-1}$.  For constants $C$, we sometimes use the convention $C(\cdots)$, if we want to emphasize the dependence of $C$ on the quantities inside the parentheses. To simplify notation in the latter sections, we write $A\lesssim B$ to mean that $A$ is bounded from above by $B$ multiplied by a uniform constant. Similarly, we say that $A\gtrsim B$, if $B\lesssim A$ and $A\approx B$ if $B\lesssim A\lesssim B$.


The \emph{upper and lower local dimensions of a measure $\mu$ at $x\in X$} are defined by
\begin{equation*}
    \updim_{\textnormal{loc}}(\mu,x)=\limsup_{r\to 0}\frac{\log \mu(B(x,r))}{\log r},
\end{equation*}
and
\begin{equation*}
    \lowdim_{\textnormal{loc}}(\mu,x)=\liminf_{r\to 0}\frac{\log \mu(B(x,r))}{\log r},
\end{equation*}
respectively. When the upper and lower limits agree, we denote the limit by $\dim_{\textnormal{loc}}(\mu,x)$ and call it the \emph{local dimension of the measure $\mu$ at $x$}. If $\updim_{\textnormal{loc}}(\mu,x)$ or $\lowdim_{\textnormal{loc}}(\mu,x)$ is a constant for $\mu$-almost every $x\in X$, we say that $\mu$ is \emph{upper or lower exact dimensional}, respectively, and similarly, if $\dim_{\textnormal{loc}}(\mu,x)$ is constant $\mu$-almost everywhere, $\mu$ is termed \emph{exact dimensional}.



\subsection{Assouad dimension of sets and connection to weak tangents}
We are mainly focused on dimensions of measures in this paper, but to place the results in a wider context, we recall the definition of the Assouad dimension of a set. One defines the \emph{Assouad dimension of $F\subset X$} by
\begin{align*}
    \adim F=\inf\bigg\{ s>0\colon& \exists C>0,\textnormal{ s.t. for all }x\in F,\; 0<r<R,\\
    &N_r(B(x,R)\cap F)\leq \left(\frac{R}{r}\right)^s\bigg\},
\end{align*}
where $N_r(E)$ denotes the smallest number of open balls of diameter $r$ needed to cover the set $E\subset X$.
A convenient way to bound the Assouad dimension of a set from below is given by the weak tangent approach. Recall that a map $T\colon X\to X$ is a \emph{similarity} if there exists $c>0$, such that
\begin{equation*}
    d(T(x),T(y))=c d(x,y),
\end{equation*}
for all $x,y\in X$. The constant $c$ is called the \emph{similarity ratio (of $T$)}.  For simplicity we give the definition of weak tangents when $X\subset \R^d$ and make a brief remark that they can be defined in complete metric spaces using pointed convergence in the Gromov-Hausdorff distance \cite{KL,MT}.  A closed set $F\subset B(0,1)$ is said to be a \emph{weak tangent} of a compact set $X\subset \R^d$ if there is a sequence of similarities $T_n\colon \R^d\to\R^d$, such that
\begin{equation*}
    T_n(X)\cap B(0,1)\to F,
\end{equation*}
in the Hausdorff distance. The collection of weak tangents of $X$ is denoted by $\mathrm{Tan}(X)$. The following proposition gives an easy way to bound the Assouad dimension from below. For the proof in the general setting see e.g. \cite[Proposition 6.1.5]{MT}.

\begin{prop}\label{prop:weaktangent}
If $X\subset \R^d$ is compact, then $\dim_{\mathrm{A}} X\geq \dim_{\mathrm{A}}F$, for all $F\in\mathrm{Tan}(X)$.
\end{prop}

\subsection{Assouad dimension of measures}
Let us now turn our attention to the dimensions of measures. Analogously to the definition for sets, one may define the \emph{(global) Assouad dimension of a measure $\mu$} fully supported on $X$ by
\begin{align*}
    \adim \mu=\inf\bigg\{ s>0\colon& \exists C>0,\textnormal{ s.t. for all }x\in X,\; 0<r<R,\\
    &\frac{\mu(B(x,R))}{\mu(B(x,r))}\leq C\left(\frac{R}{r}\right)^s\bigg\}.
\end{align*}
We sometimes use the term global Assouad dimension to avoid ambiguity with the pointwise variant we soon define.
Originally, the Assouad dimension of a measure was called the upper regularity dimension in \cite{KLV}, but due to the intimate connections between this notion of dimension and the Assouad dimension for sets, nowadays the term Assouad dimension of a measure is widely used. A simple volume argument implies that for a measure $\mu$ fully supported on a metric space $X$, we have the inequality $\adim X\leq \adim\mu$. Moreover, in \cite{VK, LS} it was shown that
\begin{equation*}
    \adim X=\inf\{\adim\mu\colon \mu\textnormal{ is a measure fully supported on }X\},
\end{equation*}
which further supports the current terminology.

The notion of the Assouad dimension of a measure is closely linked to the \emph{doubling property}, which the measure $\mu$ is said to satisfy if there is a constant $C\geq 1$, such that for any $x\in X$, $r>0$, we have
\begin{equation}\label{eq:doubling}
    \mu(B(x,2r))\leq C\mu(B(x,r)).
\end{equation}
Measures that satisfy (\ref{eq:doubling}) are called \emph{doubling measures}. In fact, it is a simple exercise to show that a measure has finite Assouad dimension if and only if it is doubling \cite[Lemma 4.1.1]{F}. In practice, the following lemma is sometimes useful when establishing the doubling property for a given measure.

\begin{lemma}\label{lemma:doubling_small_r}
Let $\mu$ be a finite Borel measure fully supported on a compact space $X$. If there are constants $C_0\geq 1,r_0>0$, such that
\begin{equation*}
    \mu(B(x,2r))\leq C_0\mu(B(x,r)),
\end{equation*}
for all $x\in X$ and $0<r<r_0$, then $\mu$ is doubling.
\end{lemma}
\begin{proof}
Let $C_0,r_0>0$ be as in the statement of the lemma and let $r\geq r_0$. Since $X$ is compact, we may cover it by a finite collection $\mathcal{B}$ of balls of radius $\frac{r_0}{3}$, and since $\mu$ is fully supported on $X$, for any $x\in X$, there is at least one $B_x\in\mathcal{B}$, such that $B_x\subset B(x,r_0)$. Let $C'=\min\{\mu(B)\colon B\in\mathcal{B}\}$. Then
\begin{equation*}
    \frac{\mu(B(x,2r))}{\mu(B(x,r))}\leq\frac{\mu(X)}{\mu(B(x,r_0))}\leq \frac{\mu(X)}{\mu(B_x)}\leq \frac{\mu(X)}{C'},
\end{equation*}
so the constant $C=\min\{\frac{\mu(X)}{C'},C_0\}$ works in (\ref{eq:doubling}).
\end{proof}

\section{Pointwise Assouad dimension}\label{sec:pw_assouad}
Motivated by the definition of the Assouad dimension, we define a pointwise variant of the Assouad dimension of a measure.

\begin{defn}\label{def:loc_assouad}
Let $\mu$ be a measure fully supported on a metric space $X$. For $x\in X$ we define the \emph{pointwise Assouad dimension of $\mu$ at $x$} as
\begin{align*}
    \dim_{\mathrm{A}}(\mu,x)=\inf\bigg\{s>0\colon &\exists C(x)>0,\textnormal{ s.t. }\forall0<r<R,\\
    &\frac{\mu(B(x,R))}{\mu(B(x,r))}\leq C(x)\left(\frac{R}{r}\right)^s\bigg\}.
\end{align*}
\end{defn}
Crucially, the constant $C$ in the Definition \ref{def:loc_assouad} may depend on the point $x$, but not on the scales $0<r<R$. Similarly as the Assouad dimension captures information on the least homogeneous points and scales of the measure, the pointwise Assouad dimension captures information on the least homogeneous scales at each point.

For the pointwise Assouad dimension, we obtain a correspondence between measures with the \emph{pointwise doubling property} and those with finite pointwise Assouad dimension. A measure $\mu$ is said to be \emph{pointwise doubling at $x\in X$}, if there is a constant $C(x)\geq 1$, such that
\begin{equation*}
    \mu(B(x,2r))\leq C(x)\mu(B(x,r)).
\end{equation*}
The following proposition collects some of the basic properties of the pointwise Assouad dimension.
\begin{prop}\label{prop:basic_properties}
Let $\mu$ be a Borel measure fully supported on a metric space $X$. Then for any $x\in X$,
\begin{enumerate}
    \item[(1)] $\dim_{\mathrm{A}}(\mu,x)$ is finite if and only if $\mu$ is pointwise doubling at $x$,
    \item[(2)] $\lowdim_{\textnormal{loc}}(\mu,x)\leq\updim_{\textnormal{loc}}(\mu,x)\leq \dim_{\mathrm{A}}(\mu,x)\leq \adim\mu$,
    \item[(3)] if $\mu$ has an atom at $x$, then $\dim_{\mathrm{A}}(\mu,x)=0$.
\end{enumerate}
\end{prop}
\begin{proof}
    Claim (1). is a trivial modification of \cite[Lemma 4.1.1]{F}. For (2), note that the first and last inequalities follow straight from the definitions, so it suffices to prove the middle inequality. Fix $x\in X$, and let $s> \dim_{\mathrm{A}}(\mu,x)$ be arbitrary. Then by definition, there is a constant $C$ depending only on $x$, such that for all $0<r<R$,
    \begin{equation*}
        \frac{\mu(B(x,r))}{\mu(B(x,R))}\geq C\left(\frac{r}{R}\right)^s.
    \end{equation*}
    In particular, by fixing $R$ we see that
    \begin{equation*}
        \mu(B(x,r))\geq c r^s,
    \end{equation*}
    where $c=\frac{C\mu(B(x,R))}{R^s}$. This implies that for $r<R$
    \begin{equation*}
        \frac{\log\mu(B(x,r))}{\log r}\leq s + \frac{\log c}{\log r},
    \end{equation*}
    and taking $r\to 0$ shows that $\updim_{\textnormal{loc}}(\mu,x)\leq s$. Since $s>\dim_{\mathrm{A}}(\mu,x)$ was arbitrary, this finishes the proof.

    For (3), assume that $\mu$ has an atom at $x\in X$. Let $0<r<R$, and note that
    \begin{equation*}
        \frac{\mu(B(x,R))}{\mu(B(x,r))}\leq \frac{\mu(X)}{\mu(\{x\})}=\frac{\mu(X)}{\mu(\{x\})}\left(\frac{R}{r}\right)^0.
    \end{equation*}
    Since the constant $\frac{\mu(X)}{\mu(\{x\})}$ depends only on $x$, we have $\dim_{\mathrm{A}}(\mu,x)\leq 0$, which is enough to prove the claim.
\end{proof}

The properties (1) and (2) of Proposition \ref{prop:basic_properties} are the pointwise analogues for the basic properties of the global Assouad dimension. Property (1) corresponds to the fact that a measure $\mu$ is doubling if and only if it has finite global Assouad dimension and property (2) to the fact that $\adim$ is ``the greatest of all dimensions'' \cite{F}, that is
\begin{equation*}
    \dim_{\mathrm{H}}\mu\leq \dim_{\mathrm{P}}\mu\leq\adim\mu,
\end{equation*}
where $\dim_{\mathrm{H}}\mu\coloneqq\essinf_{x\in X}\lowdim_{\mathrm{loc}}(\mu,x)$ and $\dim_{\mathrm{P}}\mu\coloneqq\essinf_{x\in X}\updim_{\mathrm{loc}}(\mu,x)$ are the \emph{Hausdorff dimension} and the \emph{packing dimension} of the measure $\mu$ respectively.

\begin{huom}
One can define the \emph{pointwise lower dimension of $\mu$ at $x$} analogously to the lower dimension of a measure as
\begin{align*}
    \dim_{\mathrm{L}}(\mu,x)=\sup\bigg\{s>0\colon &\exists C(x)>0,\textnormal{ s.t. }\forall0<r<R,\\
    &\frac{\mu(B(x,R))}{\mu(B(x,r))}\geq C(x)\left(\frac{R}{r}\right)^s\bigg\}.
\end{align*}
Then Proposition \ref{prop:basic_properties}(2) is strengthened to 
\begin{equation*}
    \dim_{\mathrm{L}}\mu\leq\dim_{\mathrm{L}}(\mu,x)\leq\lowdim_{\textnormal{loc}}(\mu,x)\leq\updim_{\textnormal{loc}}(\mu,x)\leq \dim_{\mathrm{A}}(\mu,x)\leq\adim\mu,
\end{equation*}
for all $x\in X$. We will not, however, pursue the study of the pointwise lower dimension any further in this paper.
\end{huom}


Next we begin investigating the relationship between the pointwise Assouad dimension and the global one. As Proposition \ref{prop:basic_properties} shows, the global Assouad dimension provides an upper bound for the pointwise Assouad dimension at every point. The natural question that arises is if the converse holds at some point, i.e. is it true that $\sup_{x\in X}\adim(\mu,x)=\adim\mu$. It turns out that generally speaking this is not the case, even in compact spaces. Recall that in Example \ref{ex:loc_not_glob_doubling}  we constructed a measure which is pointwise doubling at all points of its compact support, but fails to be globally doubling. By Proposition \ref{prop:basic_properties}(1) and the analogous fact for the global Assouad dimension, we see that this is an example of a measure with finite pointwise Assouad dimension at all points, but infinite global Assouad dimension.

After making this observation, it is natural to ask if the Assouad dimension of the supporting space is a lower bound for the pointwise Assouad dimension, as it is for the global one. Our next example shows that this is also not generally the case, in fact, there are measures supported on sets of full Assouad dimension, which have $0$ pointwise Assouad dimension at all points. The example is original, but builds on a construction by Le Donne and Rajala \cite[Example 2.20]{LR}.

\begin{example}\label{ex:loc_not_glob_doubling}
    Let $x_n=2^{-n}$, and let $\mu=\sum_{n=0}^{\infty}(3^{-n}\delta_{-x_n}+2^{-n}\delta_{x_n})$, where $\delta_x$ denotes the point mass centered at $x$. Clearly the measure is a finite Borel measure fully supported on the set $X=\{0\}\cup\bigcup_{n=0}^{\infty}\{x_n,-x_n\}$. By considering $y_k=-x_k$, and $r_k=2^{-k}$, it follows by a simple calculation that
    \begin{equation*}
        \frac{\mu(B(y_k,2r_k))}{\mu(B(y_k,r_k))}\to \infty,
    \end{equation*}
    as $k\to\infty$, which shows that $\mu$ is not doubling.
    
    The fact that $\mu$ is pointwise doubling at $X\setminus\{0\}$ follows from Proposition \ref{prop:basic_properties}(3), and a standard calculation shows that for any  $2^{-k}\leq r < 2^{-k+1}$, we have
    \begin{equation*}
        \frac{\mu(B(x,2r))}{\mu(B(x,r))}\leq \frac{2\sum_{n=k-2}^{\infty}2^{-n}}{\sum_{n=k}^{\infty}2^{-n}}\leq\frac{2^{4-k}}{2^{1-k}}=8,
    \end{equation*}
    which shows that $\mu$ is pointwise doubling at $0$.
\end{example}

\begin{example}
Let $x_{n,k}=2^{-2^n}+k4^{-2^n}$ and let $X=\{0\}\cup\bigcup_{n=1}^{\infty}\bigcup_{k=0}^{n-1}\{x_{n,k}\}$. Define the measure $\mu$ as 
\begin{equation*}
    \mu=\sum_{n=1}^{\infty}\sum_{k=0}^{n-1}\frac{2^{-n}}{n}\delta_{x_{n,k}}.
\end{equation*}
It is straightforward to show that $\mu$ is a finite doubling measure fully supported on $X$. We show that $\dim_{\mathrm{A}}X=1$, and $\dim_{\mathrm{A}}(\mu,x)=0$, for every $x\in X$.

To show that $\dim_{\mathrm{A}}X= 1$, by Proposition \ref{prop:weaktangent} it is enough to show that $[0,1]$ is a weak tangent for the set $X$. For each $n\in\N$ define a similarity $T_n:\R\to \R$ by
\begin{equation*}
    T_n(x)=n^{-1}4^{2^n}(x-2^{-2^n}),
\end{equation*}
and note that
\begin{equation*}
    T_n(X)\cap[0,1]=\bigcup_{k=0}^{n-1}\Big\{\frac{k}{n}\Big\}\to[0,1],
\end{equation*}
in the Hausdorff distance as $n\to\infty$. Thus $[0,1]$ is a weak tangent to $X$ and by Proposition \ref{prop:weaktangent} $\dim_{\mathrm{A}}X\geq 1$.

Next we show that $\dim_{\mathrm{A}}(\mu,x)=0$, for every $x\in X$. Note that every point $x\in X\setminus\{0\}$ is an atom so by Proposition \ref{prop:basic_properties}(3), $\dim_{\mathrm{A}}(\mu,x)=0$, so we only need to consider the case $x=0$. Fix $0<r<R<1$, and choose numbers $L,N\in\N$, such that
\begin{equation*}
    2^{-2^L}< r \leq 2^{-2^{L-1}}, \quad2^{-2^{N+1}}\leq R < 2^{-2^{N}},
\end{equation*}
so in particular
\begin{equation*}
    2^{L-1}\leq\frac{\log r}{\log 2},\quad 2^{N+1}\geq\frac{\log R}{\log 2}
\end{equation*}
Clearly $[0,x_{L+1,L}]\subset B(0,r)$ and $B(x,R)\subset [0,x_{N,N-1}]$, so
\begin{align*}
    \frac{\mu(B(0,R))}{\mu(B(0,r))}\leq\frac{\mu([0,x_{N,N-1}])}{\mu([0,x_{L+1,L}])}\leq 2^{L-N+1}\leq 2\frac{\log r}{\log R}.
\end{align*}
Note that for any $s>0$, the function $\phi(t)=t^s\log t$ is decreasing for  $0<t<e^{-\frac{1}{s}}$, so for all $0<r<R<e^{-\frac{1}{s}}$ we have
\begin{equation*}
    \frac{\mu(B(0,R))}{\mu(B(0,r))}\leq 2\left(\frac{R}{r}\right)^s.
\end{equation*}
Since this holds for arbitrary $s>0$, we have $\dim_{\mathrm{A}}(\mu,0)=0$.
\end{example}
In \cite{KFF}, the authors define the \emph{upper Minkowski dimension of the measure $\mu$} as
\begin{align*}
    \updim_{\mathrm{M}}\mu=\inf\{s>0\colon &\textnormal{there exists a constant }c>0\textnormal{ such that}\\
    &\mu(B(x,r))\geq cr^s,  \textnormal{ for all }x\in \text{supp}(\mu)\textnormal{ and }0<r<1\},
\end{align*}
and show in their Proposition 4.1 that when $\mu$ is compactly supported, this has the property that
\begin{equation*}
    \dim_{\mathrm{P}}\mu\leq \updim_{\mathrm{M}}\mu\leq \dim_{\mathrm{A}}\mu.
\end{equation*}
By definition of $\dim_{\mathrm{P}}\mu$ and Proposition \ref{prop:basic_properties}(2) we have the general relationship $\dim_{\mathrm{P}}\mu\leq\dim_{\mathrm{A}}(\mu,x)$, for $\mu$-almost every $x$. In general, there is no relationship between the values of $\adim(\mu,x)$ and $\updim_{\mathrm{M}}\mu$ as is evidenced by the following two examples.
\begin{example}
We give an example of a compactly supported measure $\mu$, such that
\begin{equation*}
    \updim_{\mathrm{M}}\mu>\sup_{x\in\mathrm{supp}(\mu)}\dim_{\mathrm{A}}(\mu,x).
\end{equation*}
Let $\mu=\sum_{n=0}^{\infty}(3^{-n}\delta_{-2^{-n}}+2^{-n}\delta_{2^{-n}})$ be the measure of Example \ref{ex:loc_not_glob_doubling}, which is a finite measure fully supported on $X=\{0\}\cup\bigcup_{n=0}^{\infty}\{\pm 2^{-n}\}$. Let us first show that $\updim_{\mathrm{M}}\mu\geq \frac{\log 3}{\log 2}$. Let $s<\frac{\log 3}{\log 2}$ and choose $n_0$ large enough, such that
\begin{equation*}
    \frac{n\log3}{(n+2)\log 2}> s,
\end{equation*}
for all $n\geq n_0$. Let $x_n=-2^{-n}$, and $r_n=2^{-(n+2)}$. Note that now
\begin{equation*}
    \mu(B(x_n,r_n))=3^{-n}=r_n^{\frac{n\log 3}{(n+2)\log 2}}< r_n^{s},
\end{equation*}
for all $n\geq n_0$. Since $r_n\to 0$ with $n$, this implies that $\updim_{\mathrm{M}}\mu\geq s$, and taking $s\to\frac{\log 3}{\log 2}$ gives the claim.

Note that $\mu$ has an atom at every $x\in X\setminus\{0\}$, so by Proposition \ref{prop:basic_properties}(3), $\dim_{\mathrm{A}}(\mu,x)=0<\updim_{\mathrm{M}}\mu$. To finish off the proof, we show that $\dim_{\mathrm{A}}(\mu,0)\leq1$. Fix $0<r<R$, and choose $L,N\in \N$, such that
\begin{equation}\label{eq:rR}
    2^{-L}< r \leq 2^{-L+1},\quad2^{-N-1}\leq R < 2^{-N}.
\end{equation}
Now $B(0,R)\subset [-2^{-N},2^{-N}]$, and $[-2^{L},2^{L}]\subset B(0,r)$, so by equation (\ref{eq:rR})
\begin{align*}
    \frac{\mu(B(x,R))}{\mu(B(x,r))}\leq \frac{\mu([-2^{-N},2^{-N}])}{\mu([-2^{-L},2^{-L}])}\leq\frac{2\sum_{n=N}^{\infty}2^{-n}}{\sum_{n=L}^{\infty}2^{-n}}=\frac{2^{2-N}}{2^{1-L}}\leq 8\left(\frac{R}{r}\right),
\end{align*}
and therefore $\dim_{\mathrm{A}}(\mu,0)\leq1<\frac{\log3}{\log 2}\leq \updim_{\mathrm{M}}\mu$.
\end{example}

\begin{example}
Here we give an example of a measure $\mu$, with $\updim_{\mathrm{M}}\mu<\adim(\mu,x)$, for $\mu$-almost every $x$. Let $\mu$ be a self-affine measure on a Bedford-McMullen carpet, see Section \ref{sec:ssm} for the definitions. By \cite[Theorem 8.6.2]{F}, the upper Minkowski dimension of $\mu$ is given, in the notation of Section \ref{sec:ssm}, by the formula
\begin{equation*}
    \updim_{\mathrm{M}}\mu=\max_{\bar{\imath}\in\Lambda}\left(\frac{-\log p_{\bar{\imath}}}{\log n_2}\right)+\max_{\bar{\imath}\in\Lambda}\left(\frac{\log p_1(\bar{\imath})}{\log n_2}+\frac{-\log p_1(\bar{\imath})}{\log n_1}\right),
\end{equation*}
and by our Theorem \ref{thm:bm_formula}, the pointwise Assouad dimension is given by
\begin{equation*}
    \adim(\mu,x)=\max_{\bar{\imath}\in\Lambda}\left(\frac{-\log p_{\bar{\imath}}}{\log n_2}+\frac{\log p_1(\bar{\imath})}{\log n_2}\right)+\max_{\bar{\imath}\in\Lambda}\left(\frac{-\log p_1(\bar{\imath})}{\log n_1}\right),
\end{equation*}
at $\mu$-almost every $x$. Recall that by definition,
\begin{equation*}
    p_1(\bar{\imath})=\sum_{\substack{\bar{\jmath}\in\Lambda \\i_1=j_1}}p_{\bar{\jmath}},
\end{equation*}
is the sum of the probabilities in the same column as $\bar{\imath}$. Thus by choosing the $p_{\bar{\imath}}$, for example in a way that $p_{\bar{\imath}}$ and $\frac{p_{\bar{\imath}}}{p_1(\bar{\imath})}$ are minimized in the same column, and $p_1(\bar{\imath})$ is minimized in a different column, we have
\begin{equation*}
    \adim(\mu,x)>\updim_{\mathrm{M}}\mu,
\end{equation*}
for $\mu$-almost every $x$. For example, we may choose $n_1=3$ and $n_2=4$, and $\Lambda=\{(0,0),(0,3),(2,0)\}$, with $p_{(0,0)}=\frac{1}{8},p_{(0,3)}=\frac{5}{8}$ and $p_{(2,0)}=\frac{1}{4}$. Then we have
\begin{equation*}
    \adim(\mu,x)=\frac{\log 6}{\log 4}+\frac{\log 4}{\log 3}> \frac{\log2}{\log 4}+\frac{\log 4}{\log 3}=\updim_{\mathrm{M}}\mu.,\end{equation*}
\end{example}
for $\mu$-almost every $x$.


% \begin{figure}
%     \centering
%     \begin{tikzpicture}
%     \foreach \x in {0,...,2}
%         \foreach \y in {0,...,3}
%             \draw (3*\x, 2*\y) rectangle ++(3,2);
            
%     \draw[fill=gray!40, text=black, font=\Large] (0,0) rectangle ++(3,2) node[pos=0.5]{$p_{(0,0)}=\frac{1}{8}$};
%     \draw[fill=gray!40, text=black, font=\Large] (0,6) rectangle ++(3,2) node[pos=0.5]{$p_{(0,3)}=\frac{5}{8}$};
%     \draw[fill=gray!40, text=black, font=\Large] (6,0) rectangle ++(3,2) node[pos=0.5]{$p_{(2,0)}=\frac{1}{4}$};
%     \end{tikzpicture}
%     \caption{Construction of a self-affine measure on a Bedford-McMullen carpet, with $\adim(\mu,x)>\updim_{\mathrm{M}}\mu$ at almost every $x$.}
%     \label{fig:bm-example}
% \end{figure}


\section{Quasi-Bernoulli measures on self-conformal sets}\label{sec:qb-meas}
Let $\Lambda$ be a finite index set, and associate to each $i\in\Lambda$ a contraction map $\varphi_i$ from a compact subset of $\R^d$ to itself. The collection $\{\varphi_i\}_{i\in\Lambda}$ is known as an \emph{iterated function system (IFS)}. By a foundational result of Hutchinson \cite{H}, every IFS has a unique compact and non-empty invariant set $F$ satisfying
\begin{equation*}
    F=\bigcup_{i\in\Lambda}\varphi_i(F),
\end{equation*}
called the \emph{limit set of the IFS}. To make the study easier, one often imposes some restrictions on the defining maps. In this section we will concentrate on the class of \emph{quasi-Bernoulli measures} supported on \emph{self-conformal sets}. In addition to the conformality assumption, we require that the IFS satisfies the \emph{strong separation condition (SSC)}, namely, we assume that for any distinct $i,j\in\Lambda$, we have $\varphi_i(F)\cap \varphi_j(F)=\emptyset$. Our goal is to establish an exact dimensionality property for the pointwise Assouad dimension of these measures, that is, we show that the pointwise Assouad dimension agrees with the global Assouad dimension almost everywhere.

When studying limit sets of iterated function systems, it is often useful to consider a symbolic representation of the IFS. Let $\Sigma=\{(i_1,i_2,\ldots)\colon i_k\in\Lambda\}$ denote the set of infinite sequences of the symbols in $\Lambda$. We call $\Sigma$ the \emph{symbolic space} and members of $\Sigma$ \emph{(infinite) words}. For an integer $n$, let $\Sigma_n=\{(i_1,i_2,\ldots,i_n)\colon i_k\in\Lambda\}$ be the set of \emph{finite words of length $n$} and let $\Sigma_*=\bigcup_{n\in\N}\Sigma_n\cup\{\emptyset\}$ denote the set of all finite words of any length. For any $\mathbf{i}\in\Sigma$, let $\mathbf{i}|_0=\emptyset$ denote the empty word. We use the abbreviation $\mathbf{i}=(i_1,i_2,\ldots)$ for a fixed element of $\Sigma$ and the same notation $\mathbf{i}=(i_1,\ldots,i_n)$ for elements of $\Sigma_n$, but the meaning will be clear from the context. For $\mathbf{i}=(i_1,\ldots,i_n)\in\Sigma_n$, let $\mathbf{i}^-=(i_1,\ldots,i_{n-1})$ denote the finite word obtained by dropping the last element of $\mathbf{i}$. If $\mathbf{i}\in\Sigma$, we write $\mathbf{i}|_n=(i_1,\ldots,i_n)\in\Sigma_n$ for the projection of $\mathbf{i}$ onto the first $n$ coordinates. For $\mathbf{i}\in\Sigma_n$, the cylinder $[\mathbf{i}]\subset \Sigma$ is defined to be the set of all infinite words in $\Sigma$ whose first $n$ letters are the letters of $\mathbf{i}$. In some proofs, we use for $\mathbf{i}\in\Sigma$ and $\mathbf{j}\in\Sigma_*$ the notation $\mathbf{j}\sqsubset\mathbf{i} $, to mean that the word $\mathbf{i}$ contains the word $\mathbf{j}$ as a substring.

For the contractions $\varphi_i$ we abbreviate
\begin{equation*}
    \varphi_{\mathbf{i}|_n}=\varphi_{i_1}\circ\ldots\circ \varphi_{i_n}.
\end{equation*}
Recall that there is a natural correspondence between the symbolic space $\Sigma$ and the limit set $F$ by the coding map $\pi:\Sigma\to F$ defined by
\begin{equation}\label{eq:code_map}
    \{\pi(\mathbf{i})\}=\bigcap_{n=1}^{\infty}\varphi_{\mathbf{i}|_n}(F).
\end{equation}
When $F$ satisfies the SSC, this map is a bijection.

\subsection{Self-conformal sets}\label{subsec:self-conformal}

Next we define self-conformal sets which act as a support for the measures we study in this section. Recall that an IFS $\{\varphi_i\}_{i\in\Lambda}$ on $\R^d$ is called \emph{self-conformal} if it satisfies the following assumptions:

\begin{enumerate}
    \item[(C1)] There is a set $\Omega\subset \R^d$, which is open, bounded and connected, and a compact set $X\subset\Omega$ with non-empty interior, such that
    \begin{equation*}
        \varphi_{i}(X)\subset X,
    \end{equation*}
    for all $i\in\Lambda$.
    \item[(C2)] For each $i\in\Lambda$, the map $\varphi_i$ is a $C^{1+\varepsilon}$-diffeomorphism, and $\varphi_i\colon \Omega\to\Omega$ is conformal. That is, for all $x\in\Omega$, the linear map $\varphi_i'(x)$ is a similarity. In particular, for every $y\in \Omega$, we have
    \begin{equation*}
        |\varphi_i'(x)y|= |\varphi_i'(x)||y|,
    \end{equation*}
    where $|\varphi_i'(x)|$ denotes the operator norm of the linear map $\varphi_i'(x)$.
\end{enumerate}
The use of the bounded open set $\Omega$ here is essential since contractive conformal maps defined on whole $\R^d$ are in fact similarities. The limit set $F$ of an IFS satisfying (C1) and (C2) is called a \emph{self-conformal set}. In the following we let $||\varphi'_{\mathbf{i}}||=\sup_{x\in \Omega}|\varphi'_{\mathbf{i}}(x)|$. It follows from the fact that each $\varphi_i$ is a contraction, that $||\varphi'_{\mathbf{i}}||<1$, for all $\mathbf{i}\in\Sigma_*$, and that for a fixed $\mathbf{i}\in\Sigma$, $||\varphi'_{\mathbf{i}|_n}||$ is strictly decreasing in $n$. Let us recall some key lemmas for the proof of our main theorem of this section. The first lemma can be found for example in \cite{MU}.

\begin{lemma}\label{lemma:BDP}
A self-conformal set $F$ satisfies the following.

\begin{enumerate}
    \item There exists a constant $C_1\geq 1$ such that $|\varphi_{\mathbf{i}}'(x)|\leq C_1|\varphi_{\mathbf{i}}'(y)|$, for all $\mathbf{i}\in\Sigma_*$ and $x,y\in \Omega$.
    
    \item There exists constants $C_2$ and $\delta>0$, such that for any $x,y,z\in F$, with $|x-y|\leq \delta$, we have
    \begin{equation*}
        C_2^{-1}|\varphi_{\mathbf{i}}'(z)|\leq\frac{|\varphi_{\mathbf{i}}(x)-\varphi_{\mathbf{i}}(y)|}{|x-y|}\leq C_2|\varphi_{\mathbf{i}}'(z)|,
    \end{equation*}
    for all $\mathbf{i}\in\Sigma_*$.

    \item There is a constant $C_3\geq 1$, such that
    \begin{equation*}
        C_3^{-1}\norm{\varphi'_{\mathbf{i}}}\leq\diam(\varphi_{\mathbf{i}}(F)) \leq C_3\norm{\varphi'_{\mathbf{i}}},
    \end{equation*}
    for all $\mathbf{i}\in\Sigma_*$.
    
\end{enumerate}
\end{lemma}

% For (3) note that since $F$ is compact, we may cover it by a finite collection $\{B_i\}_{i=1}^M$ of balls of radius $\delta$. Furthermore, we may assume that $B_i\cap B_{i+1}\cap F\ne\emptyset$, for all $i=1,\ldots,M-1$. Now if $x,y\in F$, we have $x\in B_{i}$ and $y\in B_{j}$ for some $i,j=1,\ldots,M$. Assume without loss of generality that $i<j$. Let $z_0=x$, $z_j=y$ and $k=i,\ldots,j-1$ pick a point $z_k\in B_{k}\cap B_{k+1}\cap F$. Then by (2), we have

% \begin{align*}
%     |\varphi_{\mathbf{i}}(x)-\varphi_{\mathbf{i}}(y)|&\leq \sum_{k=0}^j|\varphi_{\mathbf{i}}(z_k)-\varphi_{\mathbf{i}}(z_{k+1})|\\
%     &\leq C_2\norm{\varphi'_{\mathbf{i}}}\sum_{k=0}^j|z_k-z_{k+1}|\leq (M+1)\delta C_2\norm{\varphi'_{\mathbf{i}}}.
% \end{align*}
% Taking the supremum yields the upper bound with the constant $C_3=(M+1)\delta C_2$. The lower bound follows also by standard arguments, see for example \cite[(BDP.4)]{MU}
The first property in Lemma \ref{lemma:BDP} is commonly called the \emph{Bounded Distortion Property (BDP)}. The second lemma is a simple consequence of the BDP and the definition.

\begin{lemma}\label{lemma:chain}
For all $\mathbf{i},\mathbf{j}\in\Sigma_*$, we have
\begin{equation*}
    C_1^{-1}||\varphi'_{\mathbf{i}}||\cdot||\varphi_{\mathbf{j}}'||\leq||\varphi_{\mathbf{i}\mathbf{j}}'||\leq ||\varphi_{\mathbf{i}}'||\cdot||\varphi_{\mathbf{j}}'||,
\end{equation*}
where $C_1>1$ is the constant of Lemma \ref{lemma:BDP}.
\end{lemma}
\begin{proof}
Using the chain rule, and the conformality of the IFS, it is easy to see that for all $x\in F$ we have
\begin{equation*}
    |\varphi_{\mathbf{i}\mathbf{j}}'(x)|= |(\varphi_{\mathbf{i}}\circ\varphi_{\mathbf{j}})'(x)|
    =|\varphi_{\mathbf{i}}'(\varphi_{\mathbf{j}}(x))\cdot\varphi_{\mathbf{j}}'(x)|=|\varphi_{\mathbf{i}}'(\varphi_{\mathbf{j}}(x))|\cdot|\varphi_{\mathbf{j}}'(x)|.
\end{equation*}
Applying Lemma \ref{lemma:BDP} we get that for all $y\in F$
\begin{equation*}
    C_1^{-1}|\varphi_{\mathbf{i}}'(y)|\cdot|\varphi_{\mathbf{j}}'(x)|\leq |\varphi_{\mathbf{i}\mathbf{j}}'(x)| \leq |\varphi_{\mathbf{i}}'(\varphi_{\mathbf{j}}(x))|\cdot|\varphi_{\mathbf{j}}'(x)|\leq ||\varphi_{\mathbf{i}}'||\cdot||\varphi_{\mathbf{j}}'||.
\end{equation*}
The result follows by taking suprema.
\end{proof}
\begin{huom}
Let $\mathbf{i}\in\Sigma$ be $n$-periodic. Notice that, by applying the previous lemma iteratively, we have
\begin{equation*}
    C_1^{-k}||\varphi'_{\mathbf{i}|_n}||^k\leq||\varphi'_{\mathbf{i}|_{kn}}||\leq ||\varphi'_{\mathbf{i}|_n}||^k.
\end{equation*}
The exponential growth of the distortion in the lower bound is a problem in Section \ref{sec:invmeas} when we want to establish a lower bound for the Assouad dimension of the measure we investigate. The following lemma provides a precise estimate for this purpose.
\end{huom}

\begin{lemma}\label{lemma:power_equality}
If $x= \pi(\mathbf{i})$, where $\mathbf{i}\in\Sigma$ is $n$-periodic for some $n\in\N$, then for any $k\in\N$ we have
\begin{equation*}
    |\varphi_{\mathbf{i}|_{kn}}'(x)|=|\varphi_{\mathbf{i}|_{n}}'(x)|^k.
\end{equation*}
\end{lemma}
\begin{proof}
Let $\mathbf{i}\in \Sigma$ be $n$-periodic, and let $x=\pi(\mathbf{i})$. By the definition of $\pi$, this implies that 
\begin{equation}\label{eq:x_invariance}
    \varphi_{\mathbf{i}|_n}(x)=x.
\end{equation}
Let $k\in\N$. Using the chain rule, (\ref{eq:x_invariance}) and the conformality of the IFS we find that
\begin{align*}
    |\varphi_{\mathbf{i}|_{kn}}'(x)|&=|((\varphi_{i_1}\circ\ldots\circ\varphi_{i_n})\circ\underset{k\text{ times}}{\ldots}\circ(\varphi_{i_1}\circ\ldots\circ\varphi_{i_n}))'(x)|\\
    &=|(\varphi_{i_1}\circ\ldots\circ \varphi_{i_n})'(x)\cdot\underset{k\text{ times}}{\ldots}\cdot(\varphi_{i_1}\circ\ldots\circ \varphi_{i_n})'(x)|\\
    &=|\varphi_{\mathbf{i}|_n}'(x)^k|=|\varphi_{\mathbf{i}|_n}'(x)|^k.
\end{align*}
\end{proof}

\subsection{Quasi-Bernoulli measures}
Recall that a probability measure $\nu$ on $\Sigma$ is said to be \emph{quasi-Bernoulli} if there exists a uniform constant $C\geq 1$, such that for all $\mathbf{i},\mathbf{j}\in\Sigma_*$, we have
\begin{equation*}
    C^{-1}\nu([\mathbf{i}])\nu([\mathbf{j}])\leq \nu([\mathbf{i}\mathbf{j}])\leq C\nu([\mathbf{i}])\nu([\mathbf{j}]),
\end{equation*}
where $\mathbf{i}\mathbf{j}$ denotes the concatenation of the finite words $\mathbf{i}$ and $\mathbf{j}$. Note the similarity to Lemma \ref{lemma:chain}. If $C$ can be taken to equal $1$, then the measure is called a \emph{Bernoulli measure}.

Recall that two measures $\mu$ and $\nu$ are said to be \emph{equivalent} if $\mu(A)=0$ if and only if $\nu(A)=0$. In the following $\sigma\colon \Sigma\to\Sigma$ denotes the \emph{left-shift} defined by $\sigma(\iii)=i_2i_2\ldots$. The following lemma is standard, but the proof is difficult to track down from the literature so we provide a sketch here.

\begin{lemma}\label{lemma:quasi-bernoulli-equivalence}
Every quasi-Bernoulli measure is equivalent to an $\sigma$-invariant and ergodic quasi-Bernoulli measure.
\end{lemma}
\begin{proof}
Let $\nu$ be a quasi-Bernoulli measure. For all $n\in\N$ define
\begin{equation*}
    \tilde{\nu}_n=\frac{1}{n}\sum_{j=0}^{n-1}\nu\circ\sigma^{-j},
\end{equation*}
and let $\tilde{\nu}$ be the measure obtained as the weak-$*$ limit of this sequence of measures. Recall that for any $\mathbf{i}\in\Sigma_*$, the set $[\mathbf{i}]$ is open and closed and thus $\tilde{\nu}([\mathbf{i}])=\lim_{n\to\infty}\tilde{\nu}_n([\mathbf{i}])$. Using the fact that $\nu$ is quasi-Bernoulli, we have
\begin{equation*}
    \tilde{\nu}_n([\mathbf{i}])=\frac{1}{n}\sum_{j=0}^{n-1}\nu\circ\sigma^{-j}([\mathbf{i}])=\frac{1}{n}\sum_{j=0}^{n-1}\sum_{\mathbf{j}\in\Sigma_j}\nu([\mathbf{j}\mathbf{i}])\approx \frac{1}{n}\sum_{j=0}^{n-1}\underbrace{\sum_{\mathbf{j}\in\Sigma_j}\nu([\mathbf{j}])}_{=1}\nu([\mathbf{i}])=\nu([\mathbf{i}]),
\end{equation*}
for all $\mathbf{i}\in\Sigma_*$ and $n\in\N$, which shows that $\tilde{\nu}$ is quasi-Bernoulli and equivalent to $\nu$. On the other hand,
\begin{align*}
    |\tilde{\nu}([\mathbf{i}])-\tilde{\nu}(\sigma^{-1}([\mathbf{i}]))|&=\lim_{n\to\infty}|\tilde{\nu}_n([\mathbf{i}])-\tilde{\nu}_n(\sigma^{-1}([\mathbf{i}]))|\\
    &=\lim_{n\to\infty}\frac{1}{n}\left|\sum_{j=0}^{n-1}\nu\circ\sigma^{-j}([\mathbf{i}])-\sum_{j=0}^{n-1}\nu\circ\sigma^{-(j+1)}([\mathbf{i}])\right|\\
    &=\lim_{n\to\infty}\frac{1}{n}\left|\nu([\mathbf{i}])-\nu\circ\sigma^{-n}([\mathbf{i}])\right|=0.
\end{align*}
By passing to approximating cylinders, we see that this implies that $\tilde{\nu}$ is $\sigma$-invariant. 


A simple consequence of the quasi-Bernoulli property of $\nu'$ is that for all Borel sets $A,B\subset \Sigma$, and $\varepsilon>0$, there exists $n\in\N$, such that
\begin{equation*}
    \nu'(A\cap\sigma^{-n}(B))\gtrsim \nu'(A)\nu'(B)-\varepsilon.
\end{equation*}
The ergodicity of $\nu'$ follows easily from this: Let $A\subset \Sigma$ be $\sigma$-invariant and assume towards contradiction that $0<\nu'(A)<1$. Then for some $n\in\N$,
\begin{equation*}
    0=\nu'((\Sigma\setminus A) \cap A)=\nu'((\Sigma\setminus A) \cap \sigma^{-n}(A))\gtrsim\nu'(\Sigma\setminus A)\nu'(A)-\varepsilon>0,
\end{equation*}
for some small enough $\varepsilon>0$, which gives a contradiction.
\end{proof}
For the rest of this section, let $\mathcal{N}\subset \Sigma$ denote the set of infinite words, which contain all finite words as a substring, that is 
\begin{equation*}
    \mathcal{N}\coloneqq \{\mathbf{i}\in\Sigma\colon \mathbf{j}\sqsubset \mathbf{i},\textnormal{ for all }\mathbf{j\in\Sigma_*}\}.
\end{equation*}
The following lemma is simple, but crucial to the proof of Theorem \ref{thm:quasi-bernoulli-exact-assouad}.

\begin{lemma}\label{lemma:quasi-normal}
If $\nu$ is a quasi-Bernoulli measure, then $\nu(\Sigma\setminus\mathcal{N})=0$.
\end{lemma}
\begin{proof}
By Lemma \ref{lemma:quasi-bernoulli-equivalence}, we may assume without loss of generality that $\nu$ is $\sigma$-invariant and ergodic. Now for every $\mathbf{j}\in\Sigma_*$, we see by applying Birkhoff's ergodic theorem, that
\begin{align*}
    \lim_{n\to\infty}\frac{1}{n}\sum_{i=0}^{n}\chi_{[\mathbf{j}]}(\sigma^n\mathbf{i})=\int_{\Sigma}\chi_{[\mathbf{j}]}d\nu = \nu([\mathbf{j}])>0,
\end{align*}
for $\nu$-almost every $\mathbf{i}\in\Sigma$, where $\chi_{[\mathbf{j}]}$ denotes the indicator function of the set $[\mathbf{j}]$. In particular, this implies that for almost every $\mathbf{i}$, there is $n\in\N$, such that $\chi_{[\mathbf{j}]}(\sigma^n\mathbf{i})=1$, that is $\mathbf{j}\sqsubset\mathbf{i}$. Let $\Sigma_{\mathbf{j}}=\{\mathbf{i}\in\Sigma\colon \mathbf{j}\sqsubset\mathbf{i}\}$, so by the previous $\nu(\Sigma\setminus\Sigma_{\mathbf{j}})=0$. By definition of $\mathcal{N}$, we have
\begin{equation*}
    \nu(\Sigma\setminus\mathcal{N})=\nu\left(\Sigma\setminus\bigcap_{\mathbf{j}\in\Sigma_*}\Sigma_{\mathbf{j}}\right)=\nu\left(\bigcup_{\mathbf{j}\in\Sigma_*}\Sigma\setminus\Sigma_{\mathbf{j}}\right)\leq \sum_{\mathbf{j}\in\Sigma_*}\nu(\Sigma\setminus\Sigma_{\mathbf{j}})=0.
\end{equation*}
\end{proof}
We say that a measure $\mu$ supported on a self-conformal set $F$ is quasi-Bernoulli if it is the projection of a quasi-Beronulli measure $\nu$ supported on $\Sigma$ under the natural projection $\pi\colon\Sigma\to F$ defined as in (\ref{eq:code_map}). Next we show that the quasi-Bernoulli measures supported on self-conformal sets satisfying the SSC are doubling, which in particular implies that the Assouad dimensions of these measures are finite. After that, we prove the main theorem of this section, which shows that the pointwise Assouad dimension of these measures agrees with the Assouad dimension almost everywhere.
\begin{prop}\label{prop:doubling}
If $\mu$ is a quasi-Bernoulli measure fully supported on a self-conformal set $F$ satisfying the strong separation condition, then $\mu$ is doubling.
\end{prop}

\begin{proof}
Let $\delta = \min_{i\ne j}d(\varphi_i(F),\varphi_j(F))$, which is positive since $F$ is strongly separated. Fix an integer $k$ satisfying $\max_{\mathbf{i}\in\Sigma_k}\norm{\varphi_{\mathbf{i}}'}< \frac{\delta}{2C_3^2}$, where $C_3$ is the constant given by Lemma \ref{lemma:BDP}(3). Finally, let $C=\min_{\mathbf{i}\in\Sigma_{k+1}}\nu([\mathbf{i}])$.

Note that since $F$ is compact and $\mu$ is fully supported on $F$ by Lemma \ref{lemma:doubling_small_r} it suffices to consider only uniformly small values of $r>0$.
Therefore let 
\begin{equation*}
0<r<\min\{||\varphi_{\mathbf{i}|_k}'||\colon \mathbf{i}\in\Sigma_k\}.
\end{equation*}
Note that the right hand side is positive since $X$ is compact so this is possible. Also fix $x\in F$, let $\mathbf{i}\in\Sigma$ be such that $\pi(\mathbf{i})=x$, and choose $n\in\N$ as the unique integer satisfying $C_3||\varphi_{\mathbf{i}|_n}'||< r \leq C_3||\varphi_{\mathbf{i}|_{n-1}}'||$. This immediately implies that $\varphi_{\mathbf{i}|_n}(F)\subset B(x,r)$. Note that by the assumption on $r$ we have
\begin{equation*}
    ||\varphi_{\mathbf{i}|_n}'|| <||\varphi_{\mathbf{i}|_k}'||.
\end{equation*}
so in particular $n>k$. For any $l\in\N$ and $i\in\Lambda$ let $\mathbf{i}|_li$ denote the word $(i_1,i_2,\ldots, i_l,i)$ and notice that by the strong separation condition and Lemma \ref{lemma:BDP}(3), we have for all $l\in\N$ and $i\ne j$ that
\begin{equation}\label{eq:ssc}
    d(\varphi_{\mathbf{i}_l i}(F)),\varphi_{\mathbf{i}_l j}(F))\coloneqq\inf_{\substack{x\in \varphi_{\mathbf{i}_l i}(F) \\y\in \varphi_{\mathbf{i}_l j}(F)}}|x-y|\geq \delta\cdot \diam(\varphi_{\mathbf{i}|_l}(F))\geq \frac{\delta}{C_3}||\varphi_{\mathbf{i}|_l}'||.
\end{equation}
Now using Lemma \ref{lemma:chain}, we have for every $y\in B(x,2r)$
\begin{align*}
    d(y,\varphi_{\mathbf{i}|_{n-k-1}}(F))&\leq 2r< \frac{\delta}{C_3^2\max_{\mathbf{i}\in\Sigma_k}||\varphi_{\mathbf{i}}'||}r\leq \frac{\delta}{C_3\max_{\mathbf{i}\in\Sigma_k}\norm{\varphi_{\mathbf{i}}'}}||\varphi_{\mathbf{i}|_{n-1}}'||\\
    &\leq\frac{\delta}{C_3}\frac{||\varphi_{\mathbf{i}|_{n-1}}'||}{||\varphi_{\sigma^{n-k-1}\mathbf{i}|_{k}}'||}\leq \frac{\delta}{C_3} ||\varphi_{\mathbf{i}|_{n-k-1}}'||,
\end{align*}
in particular, combining this with estimate (\ref{eq:ssc}), we have $B(x,2r)\cap F\subset \varphi_{\mathbf{i}|_{n-k-1}}(F)$. Therefore, using the quasi-Bernoulli property, we have
\begin{align*}
    \frac{\mu(B(x,2r))}{\mu(B(x,r))}&\leq \frac{\mu(\varphi_{\mathbf{i}|_{n-k-1}}(F))}{\mu(\varphi_{\mathbf{i}|_{n}}(F)))}=\frac{\nu([\mathbf{i}|_{n-k-1}])}{\nu([\mathbf{i}|_{n}])}\leq \frac{1}{C_3\nu([\sigma^{n-k-1}\mathbf{i}|_{k+1}])}\leq \frac{1}{C_3C}.
\end{align*}
Since the upper bound is independent of $x$ and $r$, the claim follows.
\end{proof}
The following theorem is our main result of the section. In this generality, it is impossible to obtain an explicit value for the Assouad dimension, however, in Section \ref{sec:invmeas} we provide an example class of measures which satisfy the assumptions of the following theorem, and calculate an explicit value for their Assouad dimension.

\begin{thm}\label{thm:quasi-bernoulli-exact-assouad}
If $\mu$ is a quasi-Bernoulli measure fully supported on a self-conformal set $F$ satisfying the strong separation condition, then
\begin{equation*}
    \dim_{\mathrm{A}}(\mu,x)=\dim_{\mathrm{A}}\mu<\infty,
\end{equation*}
for $\mu$-almost every $x\in F$.
\end{thm}
\begin{proof}
It follows from Proposition \ref{prop:doubling} that $\dim_{\mathrm{A}}\mu$ is finite. Let $s<\dim_{\mathrm{A}}\mu$ and $C>0$.
Now there is a point $y\in F$ and radii $0<r<R$, satisfying
\begin{equation*}
    \frac{\mu(B(y,R))}{\mu(B(y,r))}> C\left(\frac{R}{r}\right)^s.
\end{equation*}
Let $\mathbf{i}\in \mathcal{N}$, $x=\pi(\mathbf{i})$ and let $\mathbf{j}\in\Sigma$, such that $\pi(\mathbf{j})=y$.  Now choose $k,n\in\N$ as the unique integers which satisfy
\begin{equation*}
    ||\varphi'_{\mathbf{j}|_{n+1}}||\leq R < ||\varphi'_{\mathbf{j}|_{n}}|| \text{, and }\quad ||\varphi'_{\mathbf{j}|_{k}}||< r \leq ||\varphi'_{\mathbf{j}|_{k-1}}||.
\end{equation*} Then $\varphi_{\mathbf{j}|_k}(F)\subset B(y,C_3r)$ and $B(y,\frac{\delta}{2C_3}R)\cap F\subset \varphi_{\mathbf{j}|_n}(F)$, where $C_3$ is the constant of Lemma \ref{lemma:BDP}(3). Using the quasi-Bernoulli property and the fact that $\mu$ is doubling, we get that
\begin{equation*}
    C\left(\frac{R}{r}\right)^s<\frac{\mu(B(y,R))}{\mu(B(y,r))}\lesssim \frac{\mu(\varphi_{\mathbf{j}|_n}(F))}{\mu(\varphi_{\mathbf{j}|_k}(F))}=\frac{\nu([\mathbf{j}|_n])}{\nu([\mathbf{j}|_k])}\leq C\nu([\sigma^{n}\mathbf{j}|_{k-n}])^{-1}.
\end{equation*}
Now since $\mathbf{i}\in \mathcal{N}$, there is an index $l\in\N$, such that $\sigma^l\mathbf{i}|_{k-n}=\sigma^{n}\mathbf{j}|_{k-n}$. Let $R'=||\varphi'_{\mathbf{i}|_{l}}||$ and $r'= ||\varphi'_{\mathbf{i}|_{l+n}}||$, and observe that by Lemma \ref{lemma:chain},
\begin{equation*}
    \frac{R'}{r'}=\frac{||\varphi'_{\mathbf{i}|_{l}}||}{||\varphi'_{\mathbf{i}|_{l+n}}||}\lesssim \frac{1}{||\varphi'_{\sigma^l\mathbf{i}|_{n}}||}=\frac{1}{||\varphi'_{\sigma^{n}\mathbf{j}|_{k-n}}||}\lesssim\frac{ ||\varphi'_{\mathbf{j}|_{n}}||}{||\varphi'_{\mathbf{j}|_{k}}||}=\frac{R}{r}.
\end{equation*}
Again, it is easy to see that $\varphi_{\mathbf{i}|_{l}}(F)\subset B(x,C_3R')$ and $B(x,\frac{\delta}{2C_3}r')\cap F\subset \varphi_{\mathbf{i}|_{l+n}}(F)$, so using the doubling and quasi-Bernoulli properties of $\mu$, we see that
\begin{align*}
    \frac{\mu(B(x,R'))}{\mu(B(x,r'))}&\gtrsim\frac{\mu(\varphi_{\mathbf{i}|_{l}}(F))}{\mu(\varphi_{\mathbf{i}|_{l+k-n}}(F))}=\frac{\nu([\mathbf{i}|_{l}])}{\nu([\mathbf{i}|_{l+k-n}])}\gtrsim \nu([\sigma^l\mathbf{i}|_{k-n}])^{-1}\\
    &=\nu([\sigma^{n}\mathbf{j}|_{k-n}])^{-1} \gtrsim C\left(\frac{R}{r}\right)^s\gtrsim C\left(\frac{R'}{r'}\right)^s.
\end{align*}
This shows that $\dim_{\mathrm{A}}(\mu,x)\geq s$, and taking $s\to\dim_{\mathrm{A}}\mu$ gives $\dim_{\mathrm{A}}(\mu,x)\geq \dim_{\mathrm{A}}\mu$. Since $\mathbf{i}\in\mathcal{N}$, the claim follows from Lemma \ref{lemma:quasi-normal}.
\end{proof}


\section{Measures with place dependent probabilities}\label{sec:invmeas}
In this section, we study the class of \emph{invariant measures with place dependent probabilities} supported on strongly separated self-conformal sets. The results of Section \ref{sec:qb-meas} show that these measures are doubling and that their pointwise Assouad dimension coincides with the global Assouad dimension at almost every point. Our main result of this section complements these results by giving an explicit formula for their Assouad dimension. To our knowledge, the value of the dimension is not hitherto not found in the literature. Let us begin by defining our setting.

\subsection{Place dependent invariant measures}
We assume that our IFS is self-conformal, that is it satisfies (C1) and (C2). In contrast to the case of self-conformal measures where we assign a uniform measure $p_i$ on the set $\varphi_i(F)$, now we allow the mass concentration to depend continuously on the point, that is we choose for each $i\in\Lambda$ a Hölder continuous function $p_i\colon X\to (0,1)$, which satisfy $\sum_{i\in\Lambda}p_i(x)\equiv 1$ and consider the probability  measures satisfying the equation
\begin{equation}\label{eq:pdssm}
    \int f(x)d\mu(x)=\sum_{i\in\Lambda}\int p_i(x)f\circ \varphi_i(x)d\mu(x),
\end{equation}
for $f\in C(X)$ where here and hereafter $C(X)$ denotes the set of continuous real valued functions on $X$. We define the \emph{Ruelle operator} $T\colon C(X)\to C(X)$ by
\begin{equation}\label{eq:ruelle}
    (Tf)(x)=\sum_{i\in\Lambda}p_i(x)f\circ \varphi_i(x),
\end{equation}
and let $T^*\colon M(X)\to M(X)$ denote the adjoint operator, where $M(X)$ is the set of Borel probability measures on $X$. Recall that for $\nu\in M(X)$, $T^*\nu$ is given by
\begin{equation*}
    T^*\nu(B)=\sum_{i\in\Lambda}\int_{\varphi_i^{-1}(B)}p_i(x)d\nu(x),
\end{equation*}
for all Borel subsets $B\subset X$. Barnsley et al. \cite{Barn} as well as Fan and Lau \cite{Fan} have studied the measures which are invariant under $T$ in a setting which is more general than ours. The next proposition, which is vital to this section, is a special case of \cite[Theorem 1.1]{Fan} or \cite[Theorem 2.1]{Barn} and we refer to the mentioned papers for the proof.

\begin{prop}\label{prop:invmeas}
Let $F$ be a self-conformal set satisfying the SSC and $p_i\colon X\to (0,1)$ be Hölder continuous for every $i\in\Lambda$. Then there is a unique Borel probability measure $\mu$ satisfying
\begin{equation*}
    T^*\mu=\mu.
\end{equation*}
Furthermore, for every $f\in C(X)$, $T^nf$ converges uniformly to the constant $\int f(x)d \mu(x)$.
\end{prop}
We call the measure $\mu$ an \emph{invariant measure with place dependent probabilities}, which we shorten to just \emph{invariant measure} for the remainder of this section. As was the case with self-similar measures and Bernoulli measures on the corresponding code space, there is also a natural correspondence between the invariant measure $\mu$ and a Gibbs measure on the code space. Let us define some useful notation for this section. For $\mathbf{i}\in\Sigma$ we slightly abuse notation by writing $p_i(\mathbf{i})\coloneqq p_i(\pi(\mathbf{i}))$ and $\varphi_i(\mathbf{i})\coloneqq\varphi_i(\pi(\mathbf{i}))$, where $\pi\colon \Sigma\to F$ is the natural projection given by (\ref{eq:code_map}). For $\mathbf{i}\in\Sigma$ and $n\in\N$ we let
\begin{equation*}
    p_{\mathbf{i}|_n}(\mathbf{i}) =\prod_{k=1}^{n}p_{i_k}(\sigma^{k-1}\mathbf{i}).
\end{equation*}
Denote by $P(\Sigma)\subset\Sigma$ the set of periodic points of $\Sigma$. For $\mathbf{i}\in P(\Sigma)$ with period of length $n$, we let
\begin{equation*}
    \overline{p}_{\mathbf{i}}=p_{\mathbf{i}|_{n}}(\mathbf{i}),
\end{equation*}
and
\begin{equation*}
    |\varphi'_{\mathbf{i}}|=|\varphi'_{\mathbf{i}|_{n}}(\mathbf{i})|.
\end{equation*}
The following lemma is well known consequence of the Ruelle-Perron-Frobenius theorem \cite{Bow,Fan}.

\begin{lemma}\label{lemma:gibbs}
There exists a unique $\sigma$-invariant probability measure $\nu$ on $\Sigma$, and a constant $C>1$ such that for any $x\in F$, $\mathbf{i}\in\Sigma$ and $n\in\N$, we have
\begin{equation*}
    C^{-1}p_{\mathbf{i}|_n}(\mathbf{i})\leq \nu([\mathbf{i}|_n])\leq Cp_{\mathbf{i}|_n}(\mathbf{i}).
\end{equation*}
Furthermore, $\nu$ is quasi-Bernoulli and we have $\mu=\pi_*\nu$, where $\mu$ is the measure of Proposition \ref{prop:invmeas}. 
\end{lemma}
\begin{proof}
This follows from Proposition 1.3 and Theorem 1.6 of \cite{Fan}.
\end{proof}
The previous lemma together with Proposition \ref{prop:doubling} immediately imply that the measure $\mu$ is doubling. We are now ready to state the main result of this section. Recall that $P(\Sigma)$ denotes the set of periodic words in $\Sigma$.

\begin{thm}\label{thm:pw_ssc_formula}
Let $\mu$ be a place dependent invariant measure fully supported on a self-conformal set $F$, which satisfies the SSC. Then
\begin{equation*}
    \dim_{\mathrm{A}}\mu=\sup_{\mathbf{i}\in P(\Sigma)}\frac{\log \overline{p}_{\mathbf{i}}}{\log |\varphi_{\mathbf{i}}'|}.
\end{equation*}
\end{thm}
\begin{proof}
Let us start with the upper bound. For the rest of the proof let $s=\sup_{\mathbf{i}\in P(\Sigma)}\frac{\log \overline{p}_{\mathbf{i}}}{\log |\varphi_{\mathbf{i}}'|}$. Let $x\in F$ and $\mathbf{i}\in\Sigma$, such that $\pi(\mathbf{i})=x$. Let $0<r<R$ and choose integers $k$ and $n$ which satisfy
\begin{equation*}
    |\varphi'_{\mathbf{i}|_{n+1}}(x)|\leq R < |\varphi'_{\mathbf{i}|_{n}}(x)|,\text{ and} \quad |\varphi'_{\mathbf{i}|_{k}}(x)|< r \leq |\varphi'_{\mathbf{i}|_{k-1}}(x)|.
\end{equation*}
This immediately implies that $\varphi_{\mathbf{i}|_k}(F)\subset B(x,C_3r)$, where $C_3$ is the constant of Lemma \ref{lemma:BDP}(3). As before, let $\delta = \min_{i\ne j}\{d(\varphi_i(F),\varphi_j(F))\colon i\ne j\}$. Then it is also easy to see that $B(x,\frac{\delta}{2C_3}R)\cap F\subset \varphi_{\mathbf{i}|_n}(F)$. Let us set $\mathbf{j}=(i_{k-n+1},i_{k-n+2},\ldots,i_k,i_{k-n+1},i_{k-n+2},\ldots,i_k,\ldots)\in P(\Sigma)$. Note that Lemma \ref{lemma:gibbs} shows that
\begin{equation*}
    \prod_{j=k-n+1}^kp_{i_j}(\sigma^{j-1}\mathbf{i})\gtrsim \prod_{l=1}^np_{j_l}(\sigma^{l-1}\mathbf{j}).
\end{equation*}
Using Proposition \ref{prop:doubling} to conclude that $\mu$ is doubling and Lemmas \ref{lemma:BDP}(1) and \ref{lemma:chain}, we get
\begin{align*}
    \frac{\mu(B(x,R))}{\mu(B(x,r))}&\lesssim \frac{\mu(B(x,\frac{\delta}{2C_3}R))}{\mu(B(x,C_3r))}\leq \frac{\mu(\varphi_{\mathbf{i}|_n}(F))}{\mu(\varphi_{\mathbf{i}|_k}(F))}\lesssim \frac{p_{\mathbf{i}|_n}(\mathbf{i})}{p_{\mathbf{i}|_k}(\mathbf{i})}\\
    &=\frac{\prod_{j=1}^np_{i_j}(\sigma^{j-1}\mathbf{i})}{\prod_{j=1}^kp_{i_j}(\sigma^{j-1}\mathbf{i})} =\left(\prod_{j=k-n+1}^kp_{i_j}(\sigma^{j-1}\mathbf{i})\right)^{-1}\\
    &\lesssim\left(\prod_{l=1}^np_{j_l}(\sigma^{l-1}\mathbf{j})\right)^{-1}= |\varphi'_{\mathbf{j}|_n}(\mathbf{j})|^{-\frac{\log \prod_{l=1}^np_{j_l}(\sigma^{l-1}\mathbf{j})}{\log |\varphi'_{\mathbf{j}|_n}(\mathbf{j})|}}\\
    &\leq|\varphi'_{\mathbf{j}|_n}(\mathbf{j})|^{-s}\lesssim  |\varphi'_{\mathbf{i}|_{k-n-1}}(x)|^{-s}\lesssim\left(\frac{|\varphi'_{\mathbf{i}|_{n}}(x)|}{|\varphi'_{\mathbf{i}|_{k}}(x)|}\right)^s\lesssim\left(\frac{R}{r}\right)^s.
\end{align*}
This shows that $\dim_{\mathrm{A}}\mu\leq s$, for any $x\in F$.

For the lower bound, let $t<s$, and choose $\mathbf{i}\in P(\Sigma)$, such that
\begin{equation*}
    \frac{\log \overline{p}_{\mathbf{i}}}{\log |\varphi'_{\mathbf{i}|_{n}}(x)|}\geq t,
\end{equation*}
where $x=\pi(\mathbf{i})$ and $n$ is the period of $\mathbf{i}$. For every $k\in\N$ let $r_k=|\varphi_{\mathbf{i}_{kn}}'(x)|=|\varphi_{\mathbf{i}_{n}}'(x)|^k$, where the second equality follows from Lemma \ref{lemma:power_equality}. Using the SSC and Proposition \ref{prop:doubling} we get
\begin{align*}
    \mu(B(x,r_k))&\lesssim \mu(\varphi_{\mathbf{i}|_{kn}}(F))=\prod_{j=1}^{kn}p_{i_j}(\sigma^{j-1}\mathbf{i})\\
    &=\overline{p}_{\mathbf{i}}^k=r_k^\frac{k\log \overline{p}_{\mathbf{i}}}{k\log |\varphi_{\mathbf{i}|_{n}}'(x)|}\leq r_k^t.
\end{align*}
Taking logarithms and limits shows us that
\begin{equation*}
    \updim_{\text{loc}}(\mu,x)\geq t,
\end{equation*}
so in particular by Proposition \ref{prop:basic_properties}(2),  $\dim_{\mathrm{A}}\mu\geq t$. Taking $t\to s$ finishes the proof.
\end{proof}
Using the fact that the measure $\mu$ is quasi-Bernoulli, Theorem \ref{thm:quasi-bernoulli-exact-assouad} gives the following immediate corollary.
\begin{cor}\label{cor:invmeas-pw}
If $\mu$ is a place dependent invariant measure fully supported on a self-conformal set $F$ satisfying the SSC, then
\begin{equation*}
    \dim_{\mathrm{A}}(\mu,x)=\sup_{\mathbf{i}\in P(\Sigma)}\frac{\log \Bar{p}_{\mathbf{i}}}{\log|\varphi'_{\mathbf{i}}|},
\end{equation*}
for $\mu$-almost every $x\in F$
\end{cor}

\begin{huom}\label{huom:ssm}
By \cite[Theorem 2.4]{FH}, the Assouad dimension of a self-similar measure $\mu$ under the SSC is given by the formula
\begin{equation}\label{eq:ssc_formula}
    \dim_{\mathrm{A}}\mu=\max_{i\in\Lambda}\frac{\log p_i}{\log r_i},
\end{equation}
see Section \ref{sec:ssm} for definitions. Our Theorem \ref{thm:pw_ssc_formula} can be viewed as a generalization of this result. Indeed, an IFS consisting of similarities $\varphi_i$ with similarity ratios $r_i$ is a self-conformal IFS, with $|\varphi'_i(x)|=r_i$, for all $x\in F$. Moreover, when each $p_i(x)\equiv p_i$, the assumptions of Theorem \ref{thm:pw_ssc_formula} are satisfied and a brief calculation gives us
\begin{equation*}
    \dim_{\mathrm{A}}\mu=\sup_{\mathbf{i}\in P(\Sigma)}\frac{\log \overline{p}_{\mathbf{i}}}{\log |\varphi'_{\mathbf{i}}|}=\sup_{\mathbf{i}\in \Sigma_*}\frac{\log p_{\mathbf{i}}}{\log r_{\mathbf{i}}}=\max_{i\in\Lambda}\frac{\log p_i}{\log r_i}.
\end{equation*}
This together with Corollary \ref{cor:invmeas-pw} gives the following slightly stronger version of \cite[Theorem 2.4]{FH}: If $\mu$ is a self-similar measure satisfying the SSC, then
\begin{equation*}
    \dim_{\mathrm{A}}(\mu,x)=\max_{i\in\Lambda}\frac{\log p_i}{\log r_i}=\dim_{\mathrm{A}}\mu,
\end{equation*}
for $\mu$-almost every $x\in F$.
\end{huom}

\section{Self-similar and self-affine measures}\label{sec:ssm}
In this section we will concentrate on two important IFS constructions: self-similar and self-affine measures. Let $F$ be a limit set of an IFS, as defined in the beginning of Section \ref{sec:qb-meas}, and attach to each $i\in\Lambda$ a probability $p_i\in(0,1)$, such that $\sum_{i\in\Lambda}p_i=1$. Recall that by \cite{H}, there exists a unique Borel probability measure $\mu$ fully supported on $F$ satisfying
\begin{equation*}
    \mu=\sum_{i\in\Lambda}p_i\mu\circ \varphi_i^{-1}.
\end{equation*}
When the contractions $\varphi_i$ are similarities or affine maps, $F$ is called a \emph{self-similar} or a \emph{self-affine set}, respectively, and $\mu$ is called a \emph{self-similar} or a \emph{self-affine measure}. If the maps $\varphi_i$ are similarities, we denote their similarity ratios by $r_i\in(0,1)$. In all of the proofs, we assume that $\diam(F)=1$, which does not result in loss of generality, since re-scaling the set does not affect its geometry.

Self-similar and self-affine sets and measures are perhaps the most important prototypical examples of fractal sets and measures. These classes have been well studied in the past decades, and substantial progress has been made in understanding their dimensional properties. See for example \cite{Hoch} for recent developments in the self-similar case and \cite{BHR, HR} for the self-affine case. To make their study easier, it is usual to impose some sort of separation conditions on the defining maps. The most common separation conditions are the SSC (see Section \ref{sec:qb-meas}) as well as the \emph{open set condition (OSC)}, which the set $F$ is said to satisfy if there exists an open set $U\subset \R^d$, such that $\varphi_i(U)\subset U$ for all $i\in\Lambda$ and $\varphi_i(U)\cap \varphi_j(U)=\emptyset$ for $i\ne j$. We say that a self-similar measure fully supported on a self-similar set $F$ satisfies the SSC if $F$ does and similarly for the OSC.

\subsection{Self-similar measures and the open set condition}
The doubling properties of self-similar measures are quite well studied. The fact that self-similar measures satisfying the SSC are doubling follows from Proposition \ref{prop:doubling}, and as mentioned in Remark \ref{huom:ssm}, their Assouad dimension was explicitly computed by Fraser and Howroyd \cite{FH}, and it is given by formula (\ref{eq:ssc_formula}). Relaxing the SSC to the OSC changes the situation dramatically. Yung \cite{Yung} provides examples of self-similar sets satisfying the OSC for which (1) only the canonical self-similar measure is doubling, (2) all self-similar measures are doubling, (3) the measures are doubling for some but not all choices of the weights $p_i$. In particular this shows that the Assouad dimension of self-similar measures satisfying the OSC can in many cases be infinite. Still, it is an interesting question to study the Assouad dimension of \emph{doubling} self-similar measures which do not satisfy the SSC. In the main theorem of this section, we show that if a self-similar measure satisfying the OSC is doubling, then the Assouad dimension is given by the natural formula (\ref{eq:ssc_formula}). Furthermore, we show that the pointwise Assouad dimension agrees with the global Assouad dimension almost everywhere, obtaining a stronger version of Remark \ref{huom:ssm}.

In this section, we use the notation
\begin{equation*}
    c_{\mathbf{i}|_n}=\prod_{k=1}^nc_{i_k},
\end{equation*}
for any parameters $c_i$, with $i\in\Lambda$. Recall that we may construct a Bernoulli measure $\nu$ on $\Sigma$ by setting for all $\mathbf{i}\in\Sigma_n$,
\begin{equation*}
    \nu([\mathbf{i}])=p_{\mathbf{i}},
\end{equation*}
and extending this to the whole space $\Sigma$ in the usual way. There is a natural correspondence between the Bernoulli measure $\nu$ and the self-similar measure on $F$, namely
\begin{equation}\label{eq:bernoulli}
    \mu=\pi_*\nu,
\end{equation}
where $\pi\colon \Sigma\to F$ is the natural coding map given by (\ref{eq:code_map}). The proof of the next theorem, which is our main result of this section, builds on ideas of \cite{Yung} and \cite{FH}.

\begin{thm}\label{thm:osc_formula}
Let $\mu$ be a self-similar measure satisfying the OSC. If $\mu$ is doubling, then
\begin{equation*}
    \dim_{\mathrm{A}}\mu=\max_{i\in\Lambda}\frac{\log p_i}{\log r_i},
\end{equation*}
and for $\mu$-almost every $x\in F$, we have
\begin{equation*}
    \dim_{\mathrm{A}}(\mu,x)=\adim\mu.
\end{equation*}

\end{thm}
\begin{proof}
Let $s=\max_{i\in\Lambda}\frac{\log p_i}{\log r_i}$. We start by showing that $\adim\mu\leq s$. Let $x\in F$, $0<r<R<1$ and let $\iii\in\Sigma$, be a (not necessarily unique) word satisfying $\pi(\iii)=x$. Choose integers $k$ and $l$, such that
\begin{equation*}
    r_{\mathbf{i}|_k}\leq R < r_{\mathbf{i}|_{k-1}}\text{ and }\quad r_{\mathbf{i}|_{l+1}}< r \leq r_{\mathbf{i}|_{l}}.
\end{equation*}
We may assume from this point on that $l>k$, since otherwise $\frac{R}{r}$ would be bounded from above by a uniform constant, which does not bother us. Now $\varphi_{\mathtt{i}_{l+1}}(F)\subset B(x,r)$, so in particular $\mu(B(x,r))\geq p_{\mathbf{i}|_{l+1}}$. Define $\Lambda_{x,R}=\{\mathbf{j}\in\Sigma_*\colon r_{\mathbf{j}}\leq R < r_{\mathbf{j}^{-}}, \,d(x,\varphi_{\mathbf{j}}(F))\leq R\}$. Note that the OSC implies that there is a constant $M\geq 1$ independent of $x$ and $R$, such that
\begin{equation*}
    \#\Lambda_{x,R}\leq M.
\end{equation*}
For the simple proof of this see \cite[Proposition 1.5.8]{Kig}. By definition, for every $\mathbf{j}\in\Lambda_{x,R}$ we have $\diam(\varphi_{\mathbf{j}}(F))=r_{\mathbf{j}}\leq R < r_{\mathbf{i}|_{k-1}}$,
and since $x\in \varphi_{\mathbf{i}|_{k-1}}(F)$, this implies that $d(\varphi_{\mathbf{i}|_{k-1}}(F),\varphi_{\mathbf{j}}(F))\leq R < r_{\mathbf{i}|_{k-1}}$.
Combining these estimates we see that $\varphi_{\mathbf{j}}(F)\subset B(\varphi_{\mathbf{i}|_{k-1}}(F),2r_{\mathbf{i}|_{k-1}})$, where $B(\varphi_{\mathbf{i}|_{k-1}}(F),2r_{\mathbf{i}|_{k-1}})$ denotes the open $2r_{\mathbf{i}|_{k-1}}$-neighbourhood of the set $\varphi_{\mathbf{i}|_{k-1}}(F)$. Therefore, since $\mu$ is doubling, we may apply Theorem 1.1 of \cite{Yung} to see that there is a constant $C>0$, such that
\begin{equation*}
    p_{\mathbf{j}}\leq Cp_{\mathbf{i}|_{k-1}}
\end{equation*}
holds independently from $\mathbf{j}$ and $\mathbf{i}$.
Furthermore, it is clear that
\begin{equation*}
    B(x,R)\cap F\subset\bigcup_{\mathbf{j}\in\Lambda_{x,R}}\varphi_{\mathbf{j}}(F),
\end{equation*}
so we may estimate
\begin{align*}
    \frac{\mu(B(x,R))}{\mu(B(x,r))}&\leq \frac{\sum_{\mathbf{j}\in\Lambda_{x,R}}p_{\mathbf{j}}}{p_{\mathbf{i}|_{l+1}}}\leq MC\frac{p_{\mathbf{i}|_{k-1}}}{p_{\mathbf{i}|_{l+1}}}=\frac{MC}{p_{i_k}p_{i_l}}\frac{p_{\mathbf{i}|_k}}{p_{\mathbf{i}|_l}}\\
    &\leq \frac{MC}{p_{\min}^2}\left(p_{i_{l-k+1}}p_{i_{l-k+2}}\cdots p_{i_l}\right)^{-1}\\
    &\leq \frac{MC}{p_{\min}^2}\left(r_{i_{l-k+1}}^{\frac{\log p_{i_{k-l+1}}}{\log r_{i_{k-l+1}}}}r_{i_{l-k+2}}^{\frac{\log p_{i_{k-l+2}}}{\log r_{i_{k-l+2}}}}\cdots r_{i_l}^{\frac{\log p_{i_{l}}}{\log r_{i_{l}}}}\right)^{-1}\\
    &\leq \frac{MC}{p_{\min}^2}\left(r_{i_{l-k+1}}r_{i_{l-k+2}}\cdots r_{i_l}\right)^{-s}\\&=\frac{MC}{p_{\min}^2}\left(\frac{r_{\mathbf{i}|_{k}}}{r_{\mathbf{i}|_l}}\right)^{s}
    \leq  \frac{MC}{p_{\min}^2}\left(\frac{R}{r}\right)^{s},
\end{align*}
which is enough to show that $\dim_{\mathrm{A}}\mu\leq s$.

To finish the proof, it is enough to show that the lower bound holds for the pointwise Assouad dimension at almost every point. For this, let $i\in\Lambda$ be the index maximizing $\frac{\log p_i}{\log r_i}$ and define $\mathcal{N}_n=\{\mathbf{i}\in\Sigma\colon (i,\ldots,i)\sqsubset \mathbf{i}\}$ and subsequently $\mathcal{N}=\bigcap_{n\in\N}\mathcal{N}_n$. Pick $x\in \pi(\mathcal{N})$ and note that as a special case of Lemma \ref{lemma:quasi-normal}, we have that $\pi(\mathcal{N})$ is a set of full measure. Let $\mathbf{i}\in \mathcal{N}$ be a (not necessarily unique) sequence such that $\pi(\mathbf{i})=x$. Now for any $n\in\N$ there is an integer $k$ such that
\begin{equation*}
    \mathbf{i}=(i_1,\ldots,i_k,\underbrace{i,i,\ldots,i}_n,i_{k+n+1},\ldots).
\end{equation*}
Choose $R_n = r_{\mathbf{i}|_{k}}$ and $r_n=r_{\mathbf{i}|_{k+n}}$, so $\varphi_{\smalli_{k}}(F)\subset B(x,R_n)$, and thus
\begin{equation*}
    \mu(B(x,R_n))\geq\mu(\varphi_{\smalli_{k}}(F))=p_{\mathbf{i}|_{k-1}},
\end{equation*}
and by calculations similar to above,
\begin{equation*}
    \mu(B(x,r_n))\leq MC p_{\mathbf{i}|_{k+n}}.
\end{equation*}
Therefore
\begin{align*}
    \frac{\mu(B(x,R_n))}{\mu(B(x,r_n))}\geq \frac{1}{MC}p_i^{-n}=\frac{1}{MC}(r_i^{-n})^{s}= \frac{1}{MC}\left(\frac{R_n}{r_n}\right)^{s}.
\end{align*}
Since $\frac{R_n}{r_n}\to\infty$ as $n\to \infty$, this shows that $\dim_{\mathrm{A}}(\mu,x)\geq s$. This finishes the proof, since now at $\mu$-almost every $x$, we have $s\leq \adim(\mu,x)\leq \adim\mu\leq s$.
\end{proof}

Our result generalizes \cite[Theorem 2.4]{FH} by Fraser and Howroyd, by relaxing the SSC to the OSC, with the added assumption that the measure is doubling, for obvious reasons. It is an interesting question, if the same formula (\ref{eq:ssc_formula}) for the Assouad dimension of self-similar measures works with even less restrictive separation conditions, such as the weak separation condition.

\subsection{Self-affine measures on Bedford-McMullen sponges}

A result similar to Theorem \ref{thm:quasi-bernoulli-exact-assouad} also holds for self-affine measures on very strongly separated Bedford-McMullen sponges, which we define as follows. We work in $\R^d$, with $d\geq 2$. Start by choosing integers $n_1<n_2<\ldots<n_d$, and after that choose a subset $\Lambda\subset \prod_{q=1}^d\{0,\ldots,n_q - 1\}$. The set $\Lambda$ is the code space associated with the Bedford-McMullen sponge. For all $\bar{\imath}=(i_1,i_2,\ldots,i_d)\in\Lambda$, we define an affine transform $\varphi_{\bar{\imath}}:[0,1]^d\to[0,1]^d$ by
\begin{equation*}
    \varphi_{\bar{\imath}}(x_1,\ldots,x_d)=\left(\frac{x_1+i_1}{n_1},\ldots,\frac{x_d+i_d}{n_d}\right).
\end{equation*}
The limit set of this IFS is called a \emph{Bedford-McMullen carpet} if $d=2$ or a \emph{Bedford-McMullen sponge} if $d>2$. With this construction, we associate a probability vector $(p_{\bar{\imath}})_{\bar{\imath}\in\Lambda}$, and define the self-affine measure $\mu$ on $F$ as usual. Recall that $\mu$ is related to a Bernoulli measure $\nu$ on the code space $\Sigma$ by (\ref{eq:bernoulli}). To establish bounds for the measures of balls, we need a separation condition which is strictly stronger than the strong separation condition. Following Olsen \cite{O}, we say that a Bedford-McMullen sponge $F$ satisfies the \emph{very strong separation condition (VSSC)}, if for words $(i_1,\ldots,i_d),(j_1,\ldots,j_d)\in\Lambda$ satisfying $i_k=j_k$, for all $k=1,\ldots, q-1$, and $i_q\ne j_q$, for some $q=1,\ldots,d$, we have $|i_q-j_q|>1$.
We also need the following quantity. For $q=1,\ldots,d$ and $\bar{\imath}=(i_1,\ldots,i_d)$, define
\begin{equation}\label{eq:cond_prob}
    p_q(\bar{\imath})=p(i_q|i_1,\ldots,i_{q-1})=\dfrac{\sum\limits_{\substack{\bar{\jmath}\in \Lambda\\ j_k=i_k,\,k=1,\ldots,q}}p_{\bar{\jmath}}}{\sum\limits_{\substack{\bar{\jmath}\in \Lambda\\ j_k=i_k,\,k=1,\ldots,q-1}}p_{\bar{\jmath}}},
\end{equation}
if $(i_1,\ldots,i_q,i_{q+1},\ldots,i_d)\in\Lambda$ for some $i_{q+1},\ldots,i_d$, and 0 otherwise. These numbers can be interpreted as the conditional probabilities that the $q$th digit of a randomly chosen member of $\Lambda$ equals the $q$th digit of $\bar{\imath}$, given that the first $q-1$ coordinates did. The following theorem was proved by Fraser and Howroyd \cite[Theorem 2.6]{FH}.

\begin{thm}\label{thm:bm-formula}
Let $\mu$ be a self-affine measure on a Bedford-McMullen sponge satisfying the VSSC. Then
\begin{equation*}
    \dim_{\mathrm{A}}\mu=\sum_{q=1}^d\max_{\bar{\imath}\in \Lambda}\frac{-\log p_q(\bar{\imath})}{\log n_q}.
\end{equation*}
\end{thm}
Again, we extend this result and prove that the pointwise Assouad dimension coincides with this value at almost every point.

\begin{thm}\label{thm:bm_formula}
Let $\mu$ be a self-affine measure on a Bedford-McMullen sponge $F$ satisfying the VSSC. Then
\begin{equation*}
    \dim_{\mathrm{A}}(\mu,x)=\dim_{\mathrm{A}}\mu,
\end{equation*}
for $\mu$-almost every $x\in F$.
\end{thm}
For the proof we need the concept of \emph{approximate cubes} introduced by Olsen \cite{O}. For clarity, we use $\omega$ to represent members of the set $\Sigma$ instead of $\mathbf{i}$ which we used in the self-similar case. We denote the approximate cube of level $k\in\N$ centered at $\omega=(\bar{\imath}_1,\ldots)=((i_{1,1},\ldots,i_{1,d}),\ldots)\in\Sigma$ by $Q_k(\omega)$, and define it by
\begin{equation*}
    Q_k(\omega)=\{\omega'=(\bar{\jmath}_1,\ldots)\in\Sigma\colon j_{t,q}=i_{t,q},\,\forall q=1,\ldots,d\text{ and }\forall t=1,\ldots L_q(k)\},
\end{equation*}
where $L_q(k)$ is the unique number that satisfies
\begin{equation*}
   n_q^{-L_q(k)-1} <n_1^{-k}\leq n_q^{-L_q(k)}.
\end{equation*}
In particular, this implies that
\begin{equation*}
    k\frac{\log n_1}{\log n_q}-1< L_q(k)\leq k\frac{\log n_1}{\log n_q}.
\end{equation*}
The geometric equivalent of the approximate cube $Q_k(\omega)$ is its image under the projection map $\pi:\Sigma\to \R^d$. The image $\pi(Q_k(\omega))$ is contained in
\begin{equation*}
    \prod_{q=1}^d\left[\frac{i_{1,q}}{n_q}+\ldots+\frac{i_{L_q(k),q}}{n_q^{L_q(k)}},\frac{i_{1,q}}{n_q}+\ldots+\frac{i_{L_q(k),q}}{n_q^{L_q(k)}}+\frac{1}{n_q^{L_q(k)}}\right],
\end{equation*}
which is a hypercuboid in $\R^d$ with all side lengths comparable to $n_1^{-k}$.

Olsen \cite{O} observed that the measure of an approximate cube is given by
\begin{equation}\label{eq:appr_cube_msr}
    \mu(\pi(Q_k(\omega)))=\prod_{q=1}^d\prod_{j=0}^{L_q(k)-1}p_q(\sigma^j\omega),
\end{equation}
where $\sigma:\Sigma \to \Sigma$ is the left-shift and $p_q(\omega)=p(i_{1,q}|i_{1,1},\ldots,i_{1,q-1})$, where the right hand side is as in equation (\ref{eq:cond_prob}). Recall also that a Bernoulli measure on the code space $\Sigma$ is shift invariant.

The following proposition by Olsen shows that we can approximate the balls centered at the Bedford-McMullen sponges by approximate cubes of comparable size.

\begin{prop}\label{prop:olsenvssc}
Let $\omega\in\Sigma$ and $k\in\N$.
\begin{enumerate}
    \item If the VSSC is satisfied, then $B(\pi(\omega),2^{-1}n_1^k)\cap F\subset \pi(Q_k(\omega))$.
    \item  $\pi(Q_k(\omega))\subset B(\pi(\omega),(n_1+\ldots+n_d)n_1^k)$.
\end{enumerate}
\end{prop}
The proof of the proposition can be found in \cite[Proposition 6.2.1]{O}. Let us now prove Theorem \ref{thm:bm_formula}. The proof follows ideas of Fraser and Howroyd \cite[Theorem 2.6]{FH}, where they essentially construct a single point where the pointwise Assouad dimension is maximal, which is enough to bound the Assouad dimension from below. The difficulty with establishing the lower bound for almost every point in contrast to the self-conformal case is that in addition to finding a long enough sequence of convenient symbols in the symbolic space, we also have to control its location due to the non-conformality of the sponge.

\begin{proof}[Proof of Theorem \ref{thm:bm_formula}]
First we note that $L_q(n)$ increases with $n$ and, since $n_q$ are strictly increasing, decreases with $q$. It is an elementary exercise to show that for every $k\in\N$, there is an integer $n_k$, such that for all $n\geq n_k$, we have
\begin{equation*}
    L_d(n)< L_d(n+k)<L_{d-1}(n)<L_{d-1}(n+k)<\ldots<L_1(n)<L_1(n+k).
\end{equation*}
For $q=1,\ldots,d$, let $p_q^{\min}=\min_{\bar{\imath}\in\Lambda}p_q(\bar{\imath})$, and let $\bar{\imath}_q^{\min}$ be some element of $\Lambda$ which achieves this minimum. Define for every $k\in\N$ the set
\begin{equation*}
    I_k=\bigcup_{n\geq n_k}\bigcap_{q=1}^{d}\sigma^{-L_q(n)}[\underbrace{\bar{\imath}_q^{\min},\ldots,\bar{\imath}_q^{\min}}_{L_q(n+k)-L_q(n)\text{ times}}].
\end{equation*}
Note that an element $\omega\in I_k$ has the form
\begin{align}
    \omega&=(\bar{\imath}_1,\ldots \bar{\imath}_{L_d(n)},\bar{\imath}_d^{\min},\ldots,\bar{\imath}_d^{\min},\bar{\imath}_{L_d(n+k)+1},\ldots,\bar{\imath}_{L_{2}(n)},\label{eq:omegaform}\\
    &\bar{\imath}_2^{\min},\ldots,\bar{\imath}_2^{\min}, \bar{\imath}_{L_2(n+k)+1},\ldots,\bar{\imath}_{L_1(n)},\bar{\imath}_1^{\min},\ldots,\bar{\imath}_1^{\min},\bar{\imath}_{L_1(n+k)+1},\ldots).\nonumber
\end{align}
It is also a simple exercise to show that if $\mathbf{i},\mathbf{j}\in\Sigma_*$, and $q,\ell\in \N$, such that $\ell > q+|\mathbf{i}|$, and $A,B\subset \Sigma$, with $A\subset \Lambda^q\times[\mathbf{i}]$ and $B\subset \Lambda^{\ell}\times[\mathbf{j}]$, then
\begin{equation}\label{eq:independent}
    \nu(A\cap B)=\nu(A)\nu(B).
\end{equation}
Now we choose $m_1=n_k$ and then inductively $m_i=L_1(m_{i-1}+k)+1$, for every $i>1$, and define $A_i\coloneqq\big(\bigcap_{q=1}^{d}\sigma^{-L_q(m_i)}[\underbrace{\bar{\imath}_q^{\min},\ldots,\bar{\imath}_q^{\min}}_{L_q(m_i+k)-L_q(m_i)\text{ times}}]\big)^c$. Noting that $I_k^c\subset\bigcap_{i\in\N}A_i$ and applying (\ref{eq:independent}) inductively first to the sets $A_i$ and then to the sets $\sigma^{-L_q(m_i)}[\underbrace{\bar{\imath}_q^{\min},\ldots,\bar{\imath}_q^{\min}}_{L_q(m_i+k)-L_q(m_i)\text{ times}}]$, we obtain
\begin{align*}
    \nu(I_k^c)&\leq\nu\left(\bigcap_{i\in\N}A_i\right)= \prod_{i\in\N}\nu(A_i)=\prod_{i\in\N}\Bigg(1-\nu\bigg(\bigcap_{q=1}^{d}\sigma^{-L_q(m_i)}[\underbrace{\bar{\imath}_q^{\min},\ldots,\bar{\imath}_q^{\min}}_{L_q(m_i+k)-L_q(m_i)\text{ times}}]\bigg)\Bigg)\\
    &=\prod_{i\in\N}\left(1-\prod_{q=1}^{d}(p_q^{\min})^{L_q(m_i+k)-L_q(m_i)}\right)\leq \prod_{i\in\N}\Bigg(\underbrace{1-(p_q^{\min})^d}_{<1}\Bigg)=0.
\end{align*}
Thus $\nu(I_k)=1$, and moreover
$\nu(I)=1$, where $I=\bigcap_{k\in\N}I_k$.

Now let $s=\adim\mu$ given by Theorem \ref{thm:bm-formula}, $x=\pi(\omega)$, where $\omega\in I$, and let $R_k=(n_1+\ldots+n_d)n_1^{-n-1}$, and $r_k=2^{-1}n_1^{-(n+k)-1}$, where $k$ and $n$ are chosen, such that $\omega$ is given by equation (\ref{eq:omegaform}). Observe that by Proposition \ref{prop:olsenvssc}, we have
\begin{equation*}
    \frac{\mu(B(x,R_k))}{\mu(B(x,r_k))}\geq \frac{\mu(\pi(Q_{n+1}(\omega)))}{\mu(\pi(Q_{n+k+1}(\omega)))}.
\end{equation*}
Moreover, equation (\ref{eq:appr_cube_msr}) gives
\begin{align*}
    \frac{\mu(\pi(Q_{n+1}(\omega)))}{\mu(\pi(Q_{n+k+1}(\omega)))}&=\frac{\prod_{q=1}^d\prod_{j=0}^{L_q(n)}p_q(\sigma^j\omega)}{\prod_{q=1}^d\prod_{j=0}^{L_q(n+k)}p_q(\sigma^j\omega)}=\frac{1}{\prod_{q=1}^d\prod_{j=L_q(n)-1}^{L_q(n+k)}p_q(\sigma^j\omega)}\\
    &=\prod_{q=1}^d\left(\frac{1}{p_q^{\min}}\right)^{L_q(n+k)-L_q(n)+2}\geq \prod_{q=1}^d\left(\frac{1}{p_q^{\min}}\right)^{(n+k)\frac{\log n_1}{\log n_q}-n\frac{\log n_1}{\log n_q}+1}\\
    &\geq (p_q^{\min})^{-d}\prod_{q=1}^d\left(\frac{1}{p_q^{\min}}\right)^{k\frac{\log n_1}{\log n_q}}=(p_q^{\min})^{-d}\prod_{q=1}^d\left(n_1^k\right)^{\frac{-\log p_q^{\min}}{\log n_q}}\\
    &\geq (\min_{q}p_q^{\min})^{-d}\left(n_1^k\right)^{s}=C\left(\frac{R_k}{r_k}\right)^{s},
\end{align*}
where $C=(\min_{q}p_q^{\min})^{-d}\cdot(2(n_1+\ldots+n_d))^s>0$ is a constant. Taking $k\to\infty$, we see that $\frac{R_k}{r_k}\to \infty$, which is enough to prove that $\dim_{\mathrm{A}}(\mu,x)\geq s$. This holds for all $x=\pi(\omega)$, such that $\omega\in I$, where $I$ has full measure, proving the claim.
\end{proof}

\section{Discussion}
Most of the results of this paper follow a similar pattern by providing exact dimensionality properties for the pointwise Assouad dimension. A natural follow up to the results of this paper would be to conduct finer analysis of the pointwise Assouad dimension and develop tools for \emph{multifractal analysis} of the pointwise Assouad dimension. Classically, the multifractal spectrum $f(\alpha)$ of a measure is given by the Hausdorff dimension of $\alpha$-level sets of the local dimension, that is 
\begin{equation*}
    f(\alpha)\coloneqq \dim_{\mathrm{H}}\{x\in X\colon \dim_{\mathrm{loc}}(\mu,x)=\alpha\}.
\end{equation*}
The celebrated multifractal formalism states that, in many cases, this spectrum is given by the Legendre transform of the $L^q$-spectrum of the measure, see e.g. Chapter 11 of \cite{Falc1} for details. Of course, a natural question to ask is if something similar is true for the Hausdorff dimension spectrum of the level sets of the pointwise Assouad dimension.

\begin{quest}
What is the multifractal Assouad spectrum of a strongly separated self-similar measure $\mu$? By this we mean quantity
\begin{equation*}
     f_{\mathrm{A}}(\alpha)\coloneqq \dim_{\mathrm{H}}\{x\in X\colon \adim(\mu,x)=\alpha\}.
\end{equation*}
\end{quest}
Using the Hausdorff dimension instead of the Assouad dimension in the definition is natural, since it is easy to see that each $\alpha$-level set of the pointwise Assouad dimensions is dense in the support and the Assouad dimension of sets is stable under closures.

\subsection*{Acknowledgements}
The author would like to thank Antti Käenmäki and Ville Suomala for many fruitful conversations on the contents of the paper. I also wish to express my gratitude to Balázs Bárány who introduced me to the concept of invariant measures with place dependent probabilities during his visit at the University of Oulu.

\bibliography{bibtex-refs}
\bibliographystyle{amsplain}

\end{document}