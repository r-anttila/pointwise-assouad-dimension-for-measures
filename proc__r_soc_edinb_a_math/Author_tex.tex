%% Author_tex.tex
%% V1.0
%% developed by NovaTechset
%%
%% This file describes the coding for PRM.cls

\documentclass{PRM}%%%%where PRM is the template name

%The authors can define any packages after the \documentclass{PRM} command.

%\usepackage{amsmath} for dealing with mathematics,
%\usepackage{amsthm} for dealing with theorem environments,
%\usepackage{hyperref} for linking the cross references
%\usepackage{graphics} for dealing with figures.
%\usepackage{algorithmic} for describing algorithms
%\usepackage{subfig} for getting the subfigures e.g., "Figure 1a and 1b" etc.
%\usepackage{url} It provides better support for handling and breaking URLs.

\newtheorem{theorem}{Theorem}
\newtheorem{condition}{Condition}


%The author can find the documentation of the above style file and any additional
%supporting files if required from "http://www.ctan.org"

% *** Do not adjust lengths that control margins, column widths, etc. ***

\begin{document}

\title{How to use the PRM \LaTeX\ class}

\author{\name{First author name \surname{Surname}}}

\address{First author address \email{xxxx@xxxx.xxx.xx}}

\author{Second author name \surname{Surname}}

\address{Second author address \email{xxxx@xxxx.xxx.xx}}

\begin{abstract}
This sample is a guideline for preparing technical papers using \LaTeX\
for PRM manuscript submission. It contains the documentation for PRM \LaTeX\
class file, which implements the layout of the manuscript for PRM journal.
This sample file uses a class file named \texttt{PRM.cls} where the authors
should use during their manuscript preparation.
\end{abstract}

\keywords{keyword entry 1, keyword entry 2, keyword entry 3}

\classification{35J20; 35J25; 35J50}

\maketitle

\section{Insert A head here}
This demo file is intended to serve as a ``starter file''
for PRM journal papers produced under \LaTeX\ using
PRM.cls.


\subsection{Insert B head here}
Subsection text here.


\subsubsection{Insert C head here}
Subsubsection text here.

\section{Equations}

Sample equations.

%%% Numbered equation
\begin{align}\label{1.1}
\begin{split}
\frac{\partial u(t,x)}{\partial t} = Au(t,x) \left(1-\frac{u(t,x)}{K}\right)\\
\quad -B\frac{u(t-\tau,x) w(t,x)}{1+Eu(t-\tau,x)},\\
\frac{\partial w(t,x)}{\partial t} =\delta \frac{\partial^2w(t,x)}{\partial x^2}-Cw(t,x)\\
+D\frac{u(t-\tau,x)w(t,x)}{1+Eu(t-\tau,x)},
\end{split}
\end{align}

\begin{align}\label{1.2}
\begin{split}
\frac{dU}{dt} &=\alpha U(t)(\gamma -U(t))-\frac{U(t-\tau)W(t)}{1+U(t-\tau)},\\
\frac{dW}{dt} &=-W(t)+\beta\frac{U(t-\tau)W(t)}{1+U(t-\tau)}.
\end{split}
\end{align}

%%%% Unnumbered equation
\begin{align*}
&\frac{\partial(F_1,F_2)}{\partial(c,\omega)}_{(c_0,\omega_0)} = \left|
\begin{array}{ll}
\frac{\partial F_1}{\partial c} &\frac{\partial F_1}{\partial \omega} \\\noalign{\vskip3pt}
\frac{\partial F_2}{\partial c}&\frac{\partial F_2}{\partial \omega}
\end{array}\right|_{(c_0,\omega_0)}\\
&\quad=-4c_0q\omega_0 -4c_0\omega_0p^2 =-4c_0\omega_0(q+p^2)>0.
\end{align*}

\section{Enunciations}
%%%% Most of the enunciations like theorem, lemma, corollary, proposition, defintion,
%%%% condition, example, conjecture etc. are defined in the class file.

%%%% If the author wants to add or modify the enunciation style
%%%% they can define in the preamble as shown below.

%%%% \newtheoremstyle{theorem}{6pt}{6pt}{\rm}{}{\sffamily}{ }{ }{}
%%%% \theoremstyle{theorem}
%%%% \newtheorem{theorem}{\sc Theorem}[section]

%%%%\newtheoremstyle{corollary}{6pt}{6pt}{\rm}{}{\sffamily}{ }{ }{}
%%%%\theoremstyle{corollary}
%%%%\newtheorem{corollary}{\sc Corollary}[section]

%%%%\newtheoremstyle{definition}{6pt}{6pt}{\rm}{}{\sffamily}{ }{ }{}
%%%%\theoremstyle{definition}
%%%%\newtheorem{definition}[theorem]{\sc Definition}
%%%%
%%%%\newtheorem{exercise}[theorem]{Exercise}

\begin{theorem}\label{T0.1}
Assume that $\alpha>0, \gamma>1, \beta>\frac{\gamma+1}{\gamma-1}$.
Then there exists a small $\tau_1>0$, such that for $\tau\in
[0,\tau_1)$, if $c$ crosses $c(\tau)$ from the direction of
to  a small amplitude periodic traveling wave solution of
(2.1), and the period of $(\check{u}^p(s),\check{w}^p(s))$ is
\[
\check{T}(c)=c\cdot \left[\frac{2\pi}{\omega(\tau)}+O(c-c(\tau))\right].
\]
\end{theorem}

\begin{condition}\label{C2.2}
From (0.8) and (2.10), it holds
$\frac{d\omega}{d\tau}<0,\frac{dc}{d\tau}<0$ for $\tau\in
[0,\tau_1)$. This fact yields that the system (2.1) with delay
$\tau>0$ has the periodic traveling waves for smaller wave speed $c$
than that the system (2.1) with $\tau=0$ does. That is, the
delay perturbation stimulates an early occurrence of the traveling waves.
\end{condition}

\section{Figures \& Tables}

The output for figure is:

\begin{figure}[!h]
%\centering\includegraphics[width=2.5in]{figurename.eps}
%%%call your figure name in the place "figurename.eps"
\caption{Insert figure caption here}
\label{fig_sim}
\end{figure}


%\begin{figure*}[!h]
%\centerline{\subfloat[Case I]\includegraphics[width=2.5in]{figurename.eps}%
%\label{fig_first_case}}
%\hfil
%\subfloat[Case II]{\includegraphics[width=2.5in]{figurename.eps}%
%\label{fig_second_case}}}
%\caption{Simulation results}
%\label{fig_sim}
%\end{figure*}

\vskip2pc

\noindent The output for table is:

\begin{table}[!h]
\processtable{An Example of a Table\label{table_example}}%%%Table caption goes here
{\tabcolsep10pt\begin{tabular}{|c||c|}%%%The number of columns has to be defined here
\hline
One & Two\\ %%%% Table body
\hline
Three & Four\\%%%% Table body
\hline
\end{tabular}}{}
\end{table}%%%End of the table

\section{Conclusion}
The conclusion text goes here.

\section*{Acknowledgment}

Insert the Acknowledgment text here.

\begin{thebibliography}{9}
\bibitem{bib1}
N. Ackermann. On a periodic Schr\"{o}dinger equation with nonlocal superlinear part. {\it Math.
Z}. {\bf 248} (2004), 423--443.

\bibitem{bib2}
A. Ambrosetti, M. Badiale and S. Cingolani. Semiclassical states of nonlinear Schr\"{o}dinger
equations. {\it Arch. Ration. Mech. Anal}. {\bf 140} (1997), 285--300.

\bibitem{bib3}
T. Bartch, M. Clapp and T. Weth. Configuration spaces, transfer and 2-nodal solutions of
semiclassical Sch\"{o}dinger equation. {\it Math. Ann}. {\bf 338} (2007), 147--185.

\end{thebibliography}

\appendix

\section{Appendix head}

Sample text.

\begin{align}
&\frac{\partial(F_1,F_2)}{\partial(c,\omega)}_{(c_0,\omega_0)} = \left|
\begin{array}{ll}
\frac{\partial F_1}{\partial c} &\frac{\partial F_1}{\partial \omega} \\\noalign{\vskip3pt}
\frac{\partial F_2}{\partial c}&\frac{\partial F_2}{\partial \omega}
\end{array}\right|_{(c_0,\omega_0)}\\
&\quad=-4c_0q\omega_0 -4c_0\omega_0p^2 =-4c_0\omega_0(q+p^2)>0.
\end{align}

\section{Appendix head}

Sample text.

\begin{align}
&\frac{\partial(F_1,F_2)}{\partial(c,\omega)}_{(c_0,\omega_0)} = \left|
\begin{array}{ll}
\frac{\partial F_1}{\partial c} &\frac{\partial F_1}{\partial \omega} \\\noalign{\vskip3pt}
\frac{\partial F_2}{\partial c}&\frac{\partial F_2}{\partial \omega}
\end{array}\right|_{(c_0,\omega_0)}\\
&\quad=-4c_0q\omega_0 -4c_0\omega_0p^2 =-4c_0\omega_0(q+p^2)>0.
\end{align}


\end{document}
