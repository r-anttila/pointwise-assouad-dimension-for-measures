\documentclass[12pt]{amsart}
\setcounter{secnumdepth}{3}
\numberwithin{equation}{section}

\sloppy

% PAKETIT
\usepackage{mathtools}
\usepackage[margin=1.15in]{geometry}
\usepackage{xcolor}
\usepackage{hyperref}
\usepackage{fancyhdr}
\usepackage{amsmath}             
\usepackage{amssymb}           
\usepackage{amsfonts}           
\usepackage{latexsym}           
\usepackage{amsthm}              
\usepackage{bbm}
\usepackage{tikz}

% Makroja lukujoukoille
\newcommand{\field}[1]{\mathbb{#1}}
\newcommand{\N}{\field{N}}
\newcommand{\Z}{\field{Z}}
\newcommand{\Q}{\field{Q}}
\newcommand{\R}{\field{R}}
\newcommand{\C}{\field{C}}

\newcommand{\updim}{\overline{\dim}}
\newcommand{\lowdim}{\underline{\dim}}
\newcommand{\bari}{\bar{\imath}}
\newcommand{\smalli}{\mathtt{i}}
\newcommand{\norm}[1]{\left|\left|#1\right|\right|}
\newcommand{\diam}{\mathrm{diam}}
\newcommand{\adim}{\dim_{\mathrm{A}}}

\DeclareMathOperator*{\esssup}{ess\,sup}
\DeclareMathOperator*{\essinf}{ess\,inf}

\theoremstyle{plain}
\newtheorem{thm}{Theorem}[section]   
\newtheorem{lemma}[thm]{Lemma}      
\newtheorem{cor}[thm]{Corollary}     
\newtheorem{prop}[thm]{Proposition}

\theoremstyle{definition}
\newtheorem{defn}[thm]{Definition}   
\newtheorem{quest}{Question}
\newtheorem*{quest*}{Question}

\theoremstyle{remark}
\newtheorem{huom}[thm]{Remark}   
\newtheorem{example}[thm]{Example}

\begin{document}
\title{Summary of changes to ``Pointwise Assouad dimension for measures''}
\author{Roope Anttila}

\date{\today}

\maketitle

I have implemented most of the changes suggested by the referee. The first three sections of the document received the most changes and especially the introduction got a big overhaul. The goal behind the changes to the introduction was to make it more concrete, make the main results stand out more and introduce the main definition earlier.

The referee made a comment about the shortening the paper, and the changes I have made, shortened the length of the document by four pages.

\begin{itemize}
    \item Fixed the typos and slight mistakes indicated by the referee.
    \item Moved the definitions of the Assouad dimension of the measure, the local dimensions of the measure and the pointwise Assouad dimension of the measure from the ``Preliminaries'' section to the introduction, to make the introduction more concrete.
    \item Added some discussion to the introduction about similar concepts considered in \cite{BBL}.
    \item Added Subsection 1.1 for the discussion of the main results in an attempt to make them stand out.
    \item Removed Lemma 2.3 due to its simplicity.
    \item Moved Example 2.2 to the same place as the other examples.
    \item Combined the simple results in Section 3 to a single proposition (Proposition 3.1).
    \item Moved Example 3.7 to the end of Section 6 as suggested by the referee. Also fixed the slight mistakes in the formulae for the Minkowski dimension and the Assouad dimension.
    \item Moved the examples illustrating the relationships of the pointwise Assouad dimension to their own subsection (Subsection 3.1) Also removed unnecessary steps in the calculations of the examples.
    \item Added Remark 3.2 about the definition of a pointwise lower dimension.
    \item Moved statement of the main result of Section 4 to the beginning of the section to make it stand out.
    \item Moved some definitions from Section 2 to a more natural spot in Subsection 4.2. Also added the definitions for equivalence of measures and ergodicity.
    \item Fixed the constants in the proof of Proposition 4.7.
    \item Replaced the proof of Lemma 5.2 with a reference.
    \item Clarified discussion before the proof of Theorem 6.4.
    \item Made many small changes to wording and sentence structure.
    \item Added thanks to the referee in the Acknowledgements.
\end{itemize}
\section*{Departures from suggested changes}
The referee suggested removing the proof of Lemma 4.5 and referring to existing literature. Even though the proof is quite simple, I decided against this, since it seems to be difficult to track down in the literature.

\bibliography{references}
\bibliographystyle{amsalpha}
\end{document}